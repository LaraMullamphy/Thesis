\documentclass[11pt,a4paper]{article}
\usepackage[latin1]{inputenc}
\usepackage{amsmath}
\usepackage{pdflscape}
\usepackage{rotating}
\usepackage{amsfonts}
\usepackage{tabu}
\usepackage{cite}

\numberwithin{equation}{subsection}
\usepackage[titletoc]{appendix}
\usepackage{amssymb}
\usepackage{graphicx}
\usepackage{tabularx}
\usepackage{caption}
\usepackage{multirow}
\usepackage{setspace}
\usepackage{fancyhdr}
\usepackage[export]{adjustbox}
%\usepackage{floatrow}
%\usepackage[singlelinecheck=off]{caption}
\DeclareRobustCommand\nocite[1]{%
	{\def\cite##1{\ignorespaces}#1}}
\newcommand\nocitecaption[1]{\caption[\nocite{#1}]{#1}}

% Set folder for images
\graphicspath{ {Figures_and_Images/} }

% Set Margins
\usepackage{geometry}
\geometry{, total={145mm,247mm}, left = 40mm, top = 25mm}

% Set Font
\usepackage{pslatex} %Times font

% Set up Left Align Table Caption
 \captionsetup[table]{
 	labelsep = newline, 
 	name = Table, 
 	justification=justified,
 	singlelinecheck=false,%%%%%%% a single line is centered by default
 	labelsep=colon,%%%%%%
 	skip = \bigskipamount}
 
  \captionsetup[figure]{
  	labelsep = newline, 
  	name = Figure, 
  	justification=justified,
  	singlelinecheck=true,%%%%%%% a single line is centered by default
  	labelsep=colon,%%%%%%
  	skip = \bigskipamount}


%--------------------------------------------------------------
%	TITLE PAGE
%--------------------------------------------------------------
\thispagestyle{empty}

\newcommand*{\titleGP}{\begingroup % Create the command for including the title page in the document
	\centering % Center all text
	\vspace*{\baselineskip} % White space at the top of the page
	
	
	{\Large JAMES COOK UNIVERSITY\\ [0.3\baselineskip] } % Title
	
	{\Huge COLLEGE OF SCIENCE \&{} ENGINEERING\\ [0.3\baselineskip] } % Title
	
	\vspace*{7\baselineskip} % Whitespace between location/year and editors
	
	{\Huge EG4011\\ Civil Engineering\\ [0.3\baselineskip] } % Title
	
	\vspace*{6\baselineskip} % Whitespace between location/year and editors
	
	{\Huge \bf{A STUDY OF THE EFFECTS OF NOTCHING ON RECTANGULAR AND ROUND TIMBER GIRDERS}\\ [0.3\baselineskip] } % Title
	
	\vspace*{2\baselineskip} % Whitespace between location/year and editors
	
	{\LARGE LARA MULLAMPHY\par} % Editor list

	
	\vfill % Whitespace between editor names and publisher logo
	
	
		\scshape % Small caps
		Proposal submitted to the School of Engineering and Physical Sciences in partial fulfilment of the requirements for the degree of \\
		{\Large Bachelor of Engineering\\ (Civil Engineering)} % Tagline(s) or further description
		\\[\baselineskip] % Tagline(s) or further description
		29 April 2016\par % Location and year
	
	\endgroup}

%--------------------------------------------------------------

\begin{document}

%------------------------- Title Page -------------------------
	\titleGP % This command includes the title page

	% Set Line Spacing to 1.5
	\setstretch{1.5}
	
	\pagestyle{fancy}
%	\thispagestyle{empty}
	\fancyhead{}
	\fancyfoot{}
	\renewcommand{\headrulewidth}{0pt}
	\fancyfoot[R]{\thepage}
	\pagenumbering{roman}
	\pagebreak
%-------------------- Statement of Access -----------------------
% See Thesis manual

%-------------------- Declaration of Sources -------------------

% See Thesis manual

%-------------------- Table of Contents -------------------	 
	\tableofcontents
	\vspace*{\baselineskip}
	\vspace*{\baselineskip}
	
	\pagebreak

	\nocite\listoffigures 
	\addcontentsline{toc}{section}{List of Figures}
	\listoftables
	\addcontentsline{toc}{section}{List of Tables}
	\pagebreak
	
%---------------------- Introduction --------------------------
	\section{Introduction}
	
	\pagenumbering{arabic}
	\setcounter{page}{1}
	% State the general topic

	
	\noindent
	Timber has been a dominant material in engineering and construction for centuries and continues to be a major material used in housing, floors and furniture. Timber was also the first material used to construct bridges and, although it is slowly being replaced by alternative building materials, there are currently around 20,000 timber road bridges still currently in use within Australia. Many of the bridges still in use are utilised by rail \cite{wilkinson_capacity_2008} which is subject to significant load and hence the maintenance and strengthening of these bridges is of major importance. 
    
    \vspace*{\baselineskip}
    
    \noindent
    Over the past few decades there has been a major decrease in the use of timber for the construction of bridges. This is predominately due to the inherent properties of timber which expose it to environmental problems such as weathering, rotting and insect infestation. These conditions make the bridge expensive to maintain and significantly limits the lifespan of the structure.  Hence the majority of bridges now being constructed in Queensland utilise steel and concrete to minimize the impact of any environmental damage. However, many timber bridges throughout Queensland still remain in use and hence require regular maintenance and strengthening to withstand growing load demands and remain stable until they are eventually replaced. As the complete replacement of bridges is expensive, the alternative of strengthening the structure appears to be the more feasible option. One of the primary sections of timber bridges which require strengthening are the notches in the timber beams. These sections are particularly prone to cracking and need to be studied in order to find a suitable method for strengthening these regions.
	
	\vspace*{\baselineskip}
	
	\noindent
     Notching is when a section from a member is cut out to ease insertion and act as seating for timber girders over piers or abutments. This is common practice in the construction of timber bridges and places increased risk of failure to the structure at the position of the notches.This increased risk occurs due the section loss from notching exposing the timber to the risk of cracking and potential failure. As notches reduce beam capacities, the need to strengthen this area is of high importance. Currently there is little information on notching design surrounding circular members or evaluating capacities of a notched round member. This is of significant importance in the construction of timber bridges as most members currently used in bridges are round. A study of the literature has revealed that there have been very few studies carried out on the strengthening of these notches in circular section beams. Hence there remains a strong need for further investigation, with a particular emphasis for the strengthening of notches in circular members used in the construction of timber bridges. 
	
	\vspace*{\baselineskip}
	
	\noindent
    The proposed research will provide further understanding into the effects of notches and notch angles on round timber members. Current notch design methods will be investigated and compared to determine which method obtains the most accurate results. As many bridges are currently suffering issues caused by notching, the ability to understand the effects of notching will assist in further development of  designing and possibly strengthening them. By establishing an effective method of notch design, the lifespan of current and future timber bridges may be increased and methods of maintenances can be redefined to delay the need for total replacement and overall reduce the costs associated with timber bridges.  
	
	\subsection{Objectives}
	The aim of this research was to determine the structural behaviour of rectangular and circular section timber girders with different notched angles. The main objectives were to:\par 
	
	\begin{enumerate}
		\item Determine the effects of different notch types on the flexural and shear capacity of rectangular and circular timber beams
		\item Validate existing design equations for notched timber girders
	\end{enumerate}
	
	\subsection{Scope}
	\noindent
	The overall scope of this research was to determine the effects of notching on rectangular and circular section members, and to determine which methods of notch design yield the most accurate results. This was achieved through small-scale experiments where notch angles were altered on both section types. 
	
	\pagebreak
	
	\section{Literature Review}
	
	\subsection{Timber Bridges}
	Timber has been used in the construction of bridges for thousands of years \cite{ritter_timber_1990} and remained the preferred construction material until the middle of the 20th century when it was eventually replaced by the introduction of steel and concrete \cite{ritter_timber_1990,_timber_2005} . Timber had remained the dominant building material because of its strength and light weight. Timber also has energy- absorbing properties, making it capable of handling short term overloads without harmful effects, making it an excellent material for bridge construction. Large timber members also have good fire resistance qualities, are relatively durable and have the ability to withstand de-icing effects. Overall the construction of timber bridges is very cost effective as timber is a relatively cheap and renewable resource and installation can usually be carried out without the need of heavy machinery, significantly reducing the labour required \cite{ritter_timber_1990}. Although timber bridges are relatively cost effective to build, susceptibility to weather and insects make them expensive to maintain.
	
	\vspace*{\baselineskip}
	
	\noindent
	Ongoing repairs for the timber structure can be labour intensive and expensive, hence preventative maintenance is important in cutting costs and prolonging the life of the bridge. To prevent deterioration, the timber can be chemically treated against damage to weathering from sunlight exposure and moisture. This is a major benefit as there is little to no maintenance or painting required for wood treated with preservatives \cite{ritter_timber_1990}. Furthermore, the wood can be treated to prevent pest infestation \cite{_timber_2005,ritter_timber_1990} which can cause major damage to any timber structure. The commonplace use of chemical treatment to preserve timber bridges has almost certainly added decades to the life of the bridge. However, when significant damage has already occurred, strengthening may be the only feasible option to prevent further damage and eventual failure of the bridge.
	
	\vspace*{\baselineskip}
	
	\noindent
	The availability of suitable timber is now a major concern for existing timber bridges which require substantial maintenance throughout their structural life \cite{_timber_2005}. The ability to source large sections of wood and sawn timber is becoming increasingly more difficult as the amount of timber suitable for harvesting decreases over time. The drive for forest preservation has also increased, heavily reducing the amount of suitable timber needed for bridge construction and maintenance \cite{_timber_2005,ritter_timber_1990}. This is another strong reason for strengthening current structures as the raw materials required for repair or replacement of timber bridges are costly and difficult to source.
	
	\subsubsection{Wood Types}
	There are two classes of timber; hardwood and softwood. The two types of wood can initially be distinguished by their leaves as hardwood trees have broad leaves and softwood trees have sharp needle like foliage \cite{dunningham_review_2015}. 
	
	\vspace*{\baselineskip}
	
	\noindent
	The major difference between hardwood and softwood is their cell structures \cite{stalnaker_structural_2013}. Hardwood contains mainly fibres, vessels/pores and parenchymas whereas softwoods contain tracheids (early- and latewood), resin canals and parenchymas \cite{cresswell_product_2004}.
	
	\vspace*{\baselineskip}
	
	\noindent
	It can be seen in Figure \ref{fig:Wood} that the cell structure of hardwood is substantially more complex than softwood \cite{mohanty_natural_2005}. The cell structure of softwood has a organised arrangement, whereas the cell arrangement of hard wood appears more random. A common misconception is that hardwood is hard in comparison to softwood, however this is not necessarily the case \cite{pipinato_innovative_2015,marshall_black_2005}. To determine whether to use softwood or hardwood, the major deciding factor is based on the intended purpose of the wood (i.e. whether it is to be in tension or compression).
	
		\begin{figure}[h]
			\includegraphics[scale=0.6]{Hardwoods}
			\includegraphics[scale=0.6]{Softwoods}	
			\caption{Wood Cell Structures \cite{_softwood_2016}}
			\label{fig:Wood}
		\end{figure}
	
	
	\noindent
	Table \ref{tab:Aspect} compares the aspect ratio of hardwood and softwood; these differ depending on the exact wood used and their particle/fibre size. From this it can be seen that generally hardwoods have a higher aspect ratio which implies they have better bending abilities, and thus are superior in sustaining tensile loads \cite{klyosov_wood-plastic_2007}. 
	
	\pagebreak
	\begin{center}
		\captionof{table}{Wood Cell Structures \cite{_softwood_2016}}
		\label{tab:Aspect}
		\begin{tabular}{c c c} 
			\hline
			\multirow{2}{*}{Particle size Range} & \multicolumn{2}{c}{Aspect Ratio} \\
			\cline{2-3}
			
			& Hardwoods & Softwoods \\ [0.5ex] 
			\hline
			20 mesh $(850\mu m - 0.85mm)$ & 4.6 & 3.5 \\ [0.5ex]
			
			40 mesh $(425\mu m - 0.425mm)$ & 4.4 & 3.4 \\ [0.5ex]
			
			60 mesh $(250\mu m - 0.25mm)$ & 4.4 & 4.2 \\ [0.5ex]
			
			800 mesh $(180\mu m - 0.18mm)$ & 4.2 & 4.5 \\ [0.5ex]
			
			\hline
			
		\end{tabular}
	\end{center}
	
	\vspace*{\baselineskip}
	
	\noindent
	More commonly used in timber bridge structures in Australia is hardwood due to its abundance at the time of construction\cite{rta_timber_2000}, as well its combination of high strength, durability, light-weight and most importantly flexural properties \cite{klyosov_wood-plastic_2007}. Commonly used woods in Queensland bridges are spotted gum, tally-wood and swamp mahogany. Spotted gum is strong, light-weight, tough, elastic and durable, and is particularly well-performing when in tension. Tally-wood is strong, durable and very tough; it withstands underground and aqueous conditions and is used mainly in decking and posts. Swamp mahogany is elastic, strong, tough and durable, which also sustains well in underground and aqueous conditions; it is predominately used in piles \cite{_queensland_1899}.
	
	\subsubsection{Common Defects and Failures}
    There are four major defects that are commonly occurring in timber bridges; weathering, rot- ting, cracking and termite infestation. These defects severely reduce the lifespan of a timber bridge and will all eventually lead to member failure.

    \vspace*{\baselineskip}
    
    \noindent
    \textbf{Weathering and Rotting}

	\noindent
	Weathering and rotting occur in timber members due to exposure to a combination of wind, wetting, drying and UV radiation \cite{_timber_2005,_section_2008}. The appearance of the weathering damage depends on the elements the timber is exposed to. Exposure to high wind speeds can cause dints/abrasions from small pebbles or embedded sand particles \cite{harrowfield_analysis_2006}, whereas exposure to a combination of running water and sun light can cause a ripple effect, as shown on a rail bridge longitudinal girder in the Figure \ref{fig:Weather}.
	
\begin{figure}[h]
	\begin{center}
		\includegraphics[scale=0.07]{Weathering}
	\end{center}
	\caption{Weathering}
	\label{fig:Weather}
\end{figure}
	\pagebreak
	
	 \noindent
	 Rotting occurs in areas where moisture is allowed to penetrate the wood and can lead to core rotting as shown in Figure \ref{fig:Rot} which severely reduces the strength of the member. The most vulnerable areas for rotting occur at bolt holes and cut sections (i.e. notches) where moisture can easily penetrate the timber \cite{_timber_2005,white_bridge_1992}. 
	 \vspace*{\baselineskip}
	 	\begin{figure}[h]
	 		\includegraphics[scale=0.07]{Core_Rotting}
	 		\includegraphics[scale=0.07]{Notch_Rotting}
	 		\caption{Core and Notch Rotting}
	 	    \label{fig:Rot}
	 	\end{figure} 
	
	\noindent
	Rot is decay caused by wood-destroying fungi which requires adequate moisture \cite{heckroodt_guide_2002}, heat and oxygen to prosper. This results in two different types of decay; brown rot and white rot. Brown rot is common in softwood where the fungi attacks only the cellulose creating a brown colour. However, white rot is more common in hardwood and is caused by the fungi attacking both the cellulose and lignin in the wood, creating a white colour. Rotting overall allows the rapid absorption of water and can be identified by a colour change or odor similar to anise or wintergreen\cite{white_bridge_1992}. Both weathering and rotting are effects of long-term exposure to the environment and can lead to a significant reduction in the strength of the member and potentially failure.
	
    \vspace*{\baselineskip}
	    
    \noindent
    \textbf{Insects}\par
    \noindent
    Timber is prone to two main types of insect attack; termite and lyctids. Termites require mois-ture, warm temperatures, access to their nest and usually ground contact to prosper. The most effective way to avoid termite damage is to use a timber species that is resistant to termites. Alternatively, preservative treated timber can be used or a physical or chemical barrier can be created between the timber ends and the nests \cite{_section_2008}.
	
	\vspace*{\baselineskip}
	
	\noindent
	Only the sapwood present in specific hardwoods are vulnerable to lyctids (or powder post beetles), softwoods are resistant to attack from these insects. Thus the use of softwoods or avoidance of the used of certain susceptible hardwoods will reduce the risk on lyctid attack \cite{_section_2008}.
	
	 \vspace*{\baselineskip}
	 
	\noindent
	Both these types of insects cause the same issues in timber; where they tunnel through the wooden members, causing large amounts of intricate bored out networks. This severely reduces the member strength and can eventually cause failure \cite{ryall_bridge_2001}.

	
    \vspace*{\baselineskip}
    
    \noindent
    \textbf{Cracking}
    
    \noindent
    Cracking usually occurs due to applied loads (dead and live) exceeding the strength capacities of the timber member and will eventually lead to failure. This defect can occur in any conditions and is purely dependent on the loads that are applied. However the presence of other capacity reducing defects can significantly increase the chances of cracking. The two types of cracks that occur are flexural and shear cracking \cite{_timber_2005}.
	
    \vspace*{\baselineskip}	
	
	\noindent
	Flexural cracks appear on the areas of a bridge that have high moments due to applied loads. The common location for flexural cracking are at the mid-span of the member, over the support and underneath any other permanent loads (dead loads). These areas are depicted in the Figure \ref{fig:Flex} \cite{_timber_2005}.  

 	\begin{figure}[h]
        \begin{center}
		 	\includegraphics[scale=0.8]{Failure_Location_1}
	 		\includegraphics[scale=0.8]{Failure_Location_2}
        \end{center}
	 		\caption{Locations of Flexural Cracking or Crushing \cite{_timber_2005}}
	 		\label{fig:Flex}
	\end{figure}
\pagebreak

    \noindent
	Cracking due to flexure occurs in the tensile region or face of the member and can also result in the occurrence of crushing in the compression region. Hence, flexural cracking can be a combination of tensile and compressive failure \cite{thelandersson_timber_2003}. 

    \vspace*{\baselineskip}	
    
	\noindent
	Shear cracking is caused by the shear capacity of the timber being exceeded by applied loads, resulting in horizontal cracks propagating along the grain, as shown in Figure \ref{fig:Shear}. These cracks occur in high shear stress regions throughout the bridge such as over piles and at the ends of girders \cite{_timber_2005}. 
	
	\begin{figure}[h]
 		\begin{center}
 			\includegraphics[scale=0.8]{Shear_Failure}
 		\end{center}
 		\caption{Locations of Shear Cracking \cite{_timber_2005}}
 		\label{fig:Shear}
 	\end{figure}
   \noindent 
   Both flexural and shear cracking on bridges is usually a result of bending which can cause forces in tension, compression and shear as well as moments \cite{ritter_timber_1990}. 

\subsection{Timber Properties}
    Timber is a very unique structural material and differs significantly from other man-made materials, such as steel or concrete \cite{plevris_frp-reinforced_1992}. This is due to its fibrous cell structure.
	
	\vspace*{\baselineskip}	
	
	 \noindent
	 The primary forms of failure in timber beams are tensile, shear and compressive failure. For timber, the stress-strain relationship is assumed to be uniaxial \cite{bazan_ultimate_1980,buchanan_combined_1986}, as shown in Figure \ref{fig:Stress-Strain}, where the negative region describes compression and positive region is in tension.
	 
	\begin{figure}[h]
  		\begin{center}
  			\includegraphics[scale=0.9]{Stress_Strain_Wood}
  		\end{center}
  		\caption{Standard Stress-Strain Diagram of Timber \cite{plevris_frp-reinforced_1992}}
  		\label{fig:Stress-Strain}
  	\end{figure}
	 
	 \noindent
	 The member fails in tension at a stress $f_{t}$, with a relating strain $\varepsilon_{t}$, and in compression fails at a stress $f_{c}$ and strain $\varepsilon_{c}$. The Young's modulus of the wood ($E_{w}$) is the gradient of the stress strain relationship. Beyond the compression-strain at failure ($\varepsilon_{c}$) the gradient and direction of the line changes by a constant ratio m of the Young's modulus \cite{plevris_frp-reinforced_1992,fiorelli_fiberglass-reinforced_2006}.
	 
     \vspace*{\baselineskip}	
	 
	 \noindent
	 Tensile failure is a brittle failure and is usually caused by long-term constant distributed loading. Longer members are more susceptible to tensile failure which occurs at the weakest point on the beam. Compressive failure is also caused by long-term loading and is a ductile failure \cite{thelandersson_timber_2003}. 
	 
     \vspace*{\baselineskip}
	 
   	\noindent
   	Modulus of rupture (MOR) is the maximum allowable stress a species of timber can withstand before its fibres rupture or break; which defines the bending strength \cite{markwardt_strength_1935}. It can be calculated from the maximum load the beam can carry ($F_{R}$) \cite{walker_primary_2013}  and is given by
   	
   	\begin{equation}
   	MOR = \dfrac{M}{Z} 
   	\end{equation}
   	
   	\begin{equation}
   	MOR \approx \dfrac{3 F_{R} l}{2bh^{2}} 
   	\end{equation}
   	
   	\noindent
   	where; \par
   	$ l $ = Span of the beam \par
   	$ b $ = Width of the beam \par
   	$ h $ = Height of the beam. \par
   	$ Z $ = Section modulus \par
   	 
   	\vspace*{\baselineskip}
   	
   	\noindent
   	 The modulus of elasticity (MOE) measures the flexibility or stiffness of timber and is determined through bending and deflection \cite{walker_primary_2013,agriculture_encyclopedia_2007}. The MOE in bending, also commonly denoted as $E_{b}$, can be determined from the load at the proportional limit ($F_{p}$) \cite{agriculture_encyclopedia_2007} and is given by
   	 
   	 \begin{equation}
  E_{b} \approx \dfrac{F_{p} l^{3}}{4 \Delta bh^{3}}
   	 \end{equation}
   		
   	\noindent
   	where; \par
   	$ \Delta $ = Deflection at midspan due to the load $F_{p}$.\par
   	 
   	\vspace*{\baselineskip}
	 
	\noindent 
	 The flexural ability and properties is predominately due to the aspect ratio of the wood. Aspect ratio is the defined as the ratio between the fibre length to the fibre thickness. Longer fibres result in superior mechanical properties compared to shorter fibres, thus a higher aspect ratio renders better flexural properties \cite{klyosov_wood-plastic_2007}.  
	 
	 \vspace*{\baselineskip}
	 
	 \noindent
	 The other form of failure in wood is shear failure, which is again caused by applied shear loads exceeding the shear capacity of the beam. The shear stress distribution of a standard rectangular beam can be seen in Figure \ref{fig:Shear-Stress}, and is of parabolic nature both along and perpendicular to the grain.
	 
	\begin{figure}[h]
		\begin{center}
			\includegraphics[scale=0.7]{Shear_Stress}
		\end{center}
		\caption{Shear Stress Distribution \cite{porteous_structural_2013}}
		\label{fig:Shear-Stress}
	\end{figure} 
	
	\pagebreak
	 
	 \noindent
	 The shear capacity of a timber species is a measure of the timber fibre's ability to withstand slippage. Shear capacity in timber is much greater across the grain than along the grain, where the fibres have to be broken, than along the grain where the fibres only need to be separated from each other \cite{walker_primary_2013}.
	
	
	\subsection{Australian Member Design}
	There are three common cross-sectional shapes of members in bridge construction; rectangular, circular and octagonal. The most commonly used is the rectangular section member, mainly for smaller scale constructions (e.g. frames, foot-bridges, etc.). However, for bridges requiring a large strength capacity, such as a rail bridge, the log can not be substantially reduced in size, as done to achieve rectangular cross-sections, thus the round and octagonal members are required. 
	
	
	\subsubsection{Rectangular Section}
	There are three standard Australian design codes that discuss design processes and equation for rectangular timber members; AS1720.1, AS4676 and AS4063. 

	 \vspace*{\baselineskip}
	 
	\noindent
	\textbf{Bending}\par
	% Discuss design aspects of a rectangular section member
	% Show equations and explain design process and necessary factors
	% Include all necessary tables and graphs
	
	\noindent
	The bending capacity ($M_{d}$) for an un-notched beam, with a rectangular section, is given in AS1720.1 as
	\begin{equation}
	\phi M_{d} = \phi k_{1} k_{4} k_{6} k_{9} k_{12} f'_{b} Z
	\end{equation}
	
	 where 
	 
	 $ \phi $ = Capacity Factor\par
	 
	 $ k_{1} $ = Duration of Load Factor\par
	 
	 $ k_{4} $ = Moisture Condition Factor\par
	 
	 $ k_{6} $ = Temperature Factor\par
	 
	 $ k_{9} $ = Strength Sharing Factor\par
	 
	 $ k_{12} $ = Slenderness Factor\par
	 
	 $ f'_{b} $ = Characteristic value for bending for section size
	 
	 $ Z $  = Section modulus of beam about bending axis.
	
	\vspace*{\baselineskip}
	
	\noindent
	All factors and the strength characteristic for bending can be determined from AS1720.1.
	
	\vspace*{\baselineskip}
	
	\noindent
	\textbf{Shear}\par
	
	\noindent
	The design capacity in shear of un-notched beams is described by AS1720.1 as
	\begin{equation}
	V_{d} = \phi k_{1} k_{4} k_{6} f'_{s} A_{s}
	\end{equation}
	
	\noindent
	where $\phi$, $k_{1}$, $k_{4}$, and $k_{6}$ are as determined for bending capacity design, \par
	
	$ f'_{s} $ = Characteristic value in shear\par
	
	$ A_{s} $ = Shear plane area.\par
	
	\vspace*{\baselineskip}
	
	\noindent
	and the characteristic value in shear can be found in AS1720.1. \par
	
	\vspace*{\baselineskip}
	
	\noindent
	Also when the beam is loaded about its major axis in bending, the shear plane area can be found from the breadth (b) and depth (d) using
	\begin{equation}
	A_{s} = \dfrac{2bd}{3}
	\end{equation}
	
	\subsubsection{Circular Section}
	The amount of information and research surrounding circular section members are minimal, and there are few design equations determined specifically for these members. 
	
	\vspace*{\baselineskip}
	
	\noindent
	\textbf{Bending}\par
	% Discuss design aspects of a circular section member
	% Show equations and explain design process and necessary factors
	% Include all necessary tables and graphs
	\noindent
    AS1720.1 uses the rectangular formula with additional factors to account for the change in section shape. This formula and additional factors are
	\begin{equation}
	\phi M_{d} = \phi k_{1} k_{4} k_{6} k_{9} k_{12} k_{20} k_{21} k_{22} f'_{b} Z
	\end{equation}

    \noindent
    where $\phi$, $k_{1}$, $k_{4}$, $k_{6}$, $k_{12}$ are as for working rectangular design and \par
    
    $ k_{20} $ = Immaturity Factor\par
    
    $ k_{21} $ = Shaving Factor\par
    
    $ k_{22} $ = Processing Factor.\par
	
	\vspace*{\baselineskip}
		
	\noindent
	\textbf{Shear}\par
	\noindent
	The design shear capacity given by AS1720.1 is again a modification of the rectangular design approach and is given by 
	
	\begin{equation}
	V_{d} = \phi k_{1} k_{4} k_{6} k_{20} f'_{s} A_{s}
	\end{equation}
	
	\noindent
	where the factors, $f'_{s}$ and $A_{s}$ are as specified in rectangular design, and $k_{20}$ takes into account the maturity of the material. The shear plane area for a round member is\par
	
	\begin{equation}
	A_{s} = \dfrac{3\pi d^{2}}{16}
	\end{equation}
	
	
	\noindent
	
	\subsubsection{Octagonal Section}
    Queensland Department of Transport and Main Roads (TMR) suggests that the design of such members for bending, is the same as for rectangular sections \cite{_timber_2005}. However there is very little information surrounding octagonal member design and no design methods determined specifically for this cross-section.


	\subsection{Notching}
	Notching or sniping is when the lower corner of a member is cut to make insertion easier, and to increase the stability of the member when sitting on a pier/abutment. One of the major issues faced with notches are they significantly reduce the load-carrying and shear capacities of timber beams. For these reasons, it is suggested that the use of notches be avoided in practice, however in some situations they cannot be \cite{jockwer_state---art_2013,jockwer_structural_2014}.
	
	\vspace*{\baselineskip}
	
	\noindent
    The reduction in the shear capacity of the beam causes a brittle failure and cracking to initiate from the corner of the notch, and propagate along the direction of the grain \cite{jockwer_state---art_2013,_timber_2005}. Figure \ref{fig:Notch_Crack} indicates the common behaviour of notch cracking.
    
       	\begin{figure}[h]
       		\includegraphics[scale=0.9]{Shear_Failure_Notching}
       		\caption{Notch Cracking \cite{_timber_2005}}
       		\label{fig:Notch_Crack}
       	\end{figure}
       
	\noindent
	To reduce the effects of capacity loss, TMR limit the section area loss from notching to be a maximum of 10\%. In cases where the notch is strengthened, a maximum allowable section loss can be up to 25\% \cite{_timber_2005}. As shown in Figure \ref{fig:Bending}, TMR also specifies that if there is less than 75\% of the original depth over the support, there is a high chance of failure to occur in bending. 
	
	       	\begin{figure}[h]
	       		\includegraphics[scale=0.7]{Bending_Fail}
	       		\caption{Member failure in bending over support \cite{_timber_2005}}
	       		\label{fig:Bending}
	       	\end{figure}
	
	\pagebreak
	
	\noindent
	There are three different types of fracture modes of a notch in a timber member which are dictated by the forces/stresses present at the notch corner. Figure \ref{fig:Frac_Mode} depicts these modes.

	       	\begin{figure}[h]
	       		\includegraphics[scale=0.53]{Fracture_Modes}
	       		\caption{Fracture Modes (a) mode 1  (b) mode 2  (c) mode 3 \cite{jockwer_structural_2014}}
	       		\label{fig:Frac_Mode}
	       	\end{figure}
	
	\noindent
	Fracture mode 1 is when the crack propagating from the notch opens and is caused by tensile forces at the notch corner. Mode 2 is horizontal cracking/shearing, cause by shear forces acting along the grain at the notch corner. Lastly, mode 3 is a mixed mode fracture and is caused by a combination of both tension and shear forces \cite{jockwer_structural_2014}. 
	
	\subsubsection{Notch Types}
	There are four main types of notching, rectangular end notch, tapered end notch, rounded end notch and notch in span. The figure below shows the geometry of each type of notch.
	
	\begin{center}
		\begin{figure}[h]
			\includegraphics[scale=0.9]{Notch_Types}
			\caption{Notching Types; (a) rectangular end notch; (b) tapered end notch; (c) rounded end notch; (d) notch in span \cite{jockwer_state---art_2013}}
		\end{figure}
	\end{center}
	

	\noindent
	There is little information surrounding the exact effects of each type of notch on members. However, it is recommended by Australian Standard AS1720.1 that a tapered notch with a 1:4 gradient chamfer from the notch corner be used if notching is required. In theory, this particular tapered notch will increase the member's shear capacity by three times in comparison to a rectangular end notch\cite{_timber_2005}. 
	
	
	\subsubsection{Notch Design}	
	There are many different methods for designing notched members, based on significantly different concepts and, in most cases, obtaining varied results \cite{jockwer_structural_2014}. There are few methods in designing round end notches and notches within member spans, and most standards only design for a rectangular or tapered notch. However, the greatest limitation in notch design is all methods are based on rectangular section beams and do not account for circular or octagonal sections \cite{_timber_2005}. This is a major issue, as round and octagonal sections are preferred in timber bridges, due to their larger sections. 
	
	\vspace*{\baselineskip}
	
	\noindent
	\textbf{Australian Standards}\par
	\noindent
	The notched member design method used in AS1720.1 focuses on stress intensities and accounts for the effects of the maximum bending moment ($M^{*}$) and maximum shear force ($V^{*}$) occurring at the corner of the notch; where any negative values for bending moment or shear are neglected  \cite{jockwer_structural_2014}. AS1720.1 notch design equation is
	
	\begin{equation}
    \dfrac{6M^*}{bd_n^2} + \dfrac{6V^*}{bd_n} \leq \phi g_{40} k_1 k_4 k_6 k_{12} f'_{sj}
    \label{eq:Aus}
	\end{equation}
	
	\noindent
	where \par
	$ f'_{sj} $ = Shear strength\par
	$ g_{40} $ = Notch coefficient.\par
	
	\vspace*{\baselineskip}
	
	\noindent
	The k-factors and $\phi$ are purely dependent on location, use and specific timber characteristics, and are found as per usual design methods in AS1720.1. The other variables used in equation \ref{eq:Aus} are depicted in Figure \ref{fig:figure2}.
	
	\begin{center}
		\begin{figure}[h]
			\includegraphics[scale=0.9]{Notching.png}
			\caption{Notation for Notch}
			\label{fig:figure2}
		\end{figure}
	\end{center}
	
	\pagebreak
	
	\noindent
	The notch coefficient g40 depends on the taper and height of the notch and can be found according to Table \ref{tab:g40}.
	
	\begin{center}
		\captionof{table}{Notch Coefficient}
		\label{tab:g40}
		\begin{tabularx}{\textwidth}{>{\centering}X|>{\centering}X|>{\centering}X} 
			\hline \hline
			\multirow{2}{*}{\textbf{Notch Angle Slope}} & \multicolumn{2}{c}{$g_{40}$} \\
			\cline{2-3}
			
			&$d_{notch} \geq 0.1d$ & $d_{notch} < 0.1d$ \tabularnewline [0.5ex] 
			\hline
			$l_{notch}/d_{notch}=0$ & $9.0/d^{0.45}$ & $3.2/d^{0.45}_{notch}$ \tabularnewline [0.5ex]
			\hline
			$l_{notch}/d_{notch}=2$ & $9.0/d^{0.33}$ & $4.2/d^{0.33}_{notch}$ \tabularnewline [0.5ex]
			\hline
			$l_{notch}/d_{notch}=4$ & $9.0/d^{0.24}$ & $5.2/d^{0.24}_{notch}$ \tabularnewline [0.5ex]
			\hline \hline
		\end{tabularx}
	\end{center}
	
	\vspace*{\baselineskip}
	
    \noindent
	AS1720.1 specifies that for this design method, no strength-reducing characteristics, such as knots, are allowed within 150mm from the notch corner.
	\vspace*{\baselineskip}
		
	\noindent
	Australian design method is based on LEFM, which is basic fracture mechanic concepts under the assumptions that the material is linearly elastic. However this assumption is not ideal for timber, as it's elastic properties are considered orthotropic. 
	
	\vspace*{\baselineskip}
	
	\noindent 	
	\textbf{Fracture Mechanics}\par 	 	
	\noindent 	
	The AS1720.1 notched member design method considers only the effect of stress intensities and does not take into account fracture mechanics. In 1988, Gustafsson proposed an equation to design for strength of the notch, based on fracture energy \cite{jockwer_state---art_2013,gustafsson_study_1988}.  	 \begin{equation} 
	\dfrac{V_{f}}{b\alpha d} = \dfrac{\sqrt{\frac{G_{c}}{d}}}{\sqrt{0.6\frac{(\alpha-\alpha^{2})}{G_{xy}}}+\beta\sqrt{6\frac{(1/\alpha-\alpha^2)}{E_{x}}}} \end{equation} 		 		 		
	\noindent 		
	Where; \par 	
	$ V_{f} $ - Shear force at fracture of notch\par 	
	$G_{xy}$ - Shear modulus\par  	
	$ E_{x} $ - Modulus of elasticity in beam direction, parallel to the grain\par 	
	$ d $ - depth of the member\par  	
	$ G_{c} $ - Fracture energy\par  	
	$ \beta $, 
	$ \alpha $ and h are depicted in the figure below. \par 		 	
	\vspace*{\baselineskip} 	 	
	\noindent 	This equation was established for a beam notched at both ends, and in accordance with the characteristics shown in the figure \ref{fig:Gust}. The actions due to the moment and shear, along with the effect from elastic clamping, results in a deflection, which is the basis of this fracture energy approach. Gustafsson derived this deflection, also considering lower bending stiffness of the cross-section at the junction of the notched part of the beam, and assumed its proportionality to the moment acting at the notch corner \cite{jockwer_state---art_2013}. Clamping was also incorporated with a factor, which was conservatively chosen as $1/\alpha^{3}$ \cite{gustafsson_study_1988}.   
		 	 		
		\begin{figure}[h] 			\includegraphics[scale=0.9]{Geom_notched_beam_dowl} 			\caption{Geometry of notched beam and dowel joint \cite{serrano_fracture_2007}} 	
		\label{fig:Gust}	
		\end{figure} 	
	 		 	
	\noindent 	
	The fracture energy for a notched member can be found using the following equation \cite{serrano_fracture_2007}.  	 	
	\begin{equation} 		
	G_{c} = \dfrac{1}{2}(P_{f}^{2}/B) \dfrac{dC(a)}{da}  	\end{equation} 	 	
	Where;\par 	
	C - denotes the compliance of the structure\par 	
	da - extension of the length of the crack\par 	
	$P_{f}$ - magnitude of load that start propagation of the crack\par 	
	B - width of the fracture area (i.e. $B = dA/da$) 
	   
	 \vspace*{\baselineskip} 	 	
	 
	 \noindent 
	 Similar notch strength equations have been derived (Smith et al. 1996), with a defining difference of the clamping effect. Smith et al proposed a notch strength relationship from the same fracture mechanic concepts, but taking a clamping factor of $1/\alpha^{2}$. It has been found through comparison, that the prediction of the notch strength proposed by Gustafsson was more conservative and accurate than Smith et al. which has given results exceeding the notches capacity \cite{jockwer_state---art_2013}.   	
	 
	 \vspace*{\baselineskip} 	 	
	 
	 \noindent 	This approach is used as a verification of the shear stress in the notch cross-section and has become a basis for design methods in European and Canadian design codes.
	
	\vspace*{\baselineskip}
	
	\noindent
	\textbf{Eurocode 5 Approach}\par
	
	\noindent
	The European timber structure design code (Eurocode 5) is based on studies of notch fracture mechanics and specifies the following equation for notch design as
	 
	\begin{equation}
	\tau_{d} = 1.5\dfrac{V}{b h_{ef}} \leq k_{v} f_{v,d} 
	\label{eq:Euro}
	\end{equation}
     
     where\par
     $ \tau_{d} $ = Shear stress \par
     $ V $ = Shear force \par
     $ h_{ef} $ = Depth above notch \par    
     
     \vspace*{\baselineskip} 
     
     \noindent
     The shear strength, $ f_{v,d} $, can be found using
     
   	\begin{equation}
   	f_{v,d} = \dfrac{k_{mod} f_{v,k}}{\gamma M} 
   	\end{equation} 
     
     \noindent
     where $f_{v,k}$ is the characteristic shear strength, $k_{mod}$ is a strength modification factor and $\gamma M$ is a material modification factor, all of which can be found in Eurocode 5 \cite{_eurocode_1995}.
     
     \vspace*{\baselineskip} 
     
     \noindent
     The notch factor $k_{v}$ is derived from Gustafsson's 1988 equation \cite{gustafsson_study_1988}, as well as modifications to account for the taper of the notch ($i$) established by Riberholt \cite{riberholt_timber_1991} and is given by
     
     \begin{equation}
     k_{v} = min
     \begin{cases}
	 1  \\ 
	 \frac{k_{n}(1+\frac{1.1i^{1.5}}{\sqrt{h}})}{\sqrt{h}(\sqrt{\alpha(1-\alpha)} + 0.8\frac{x}{h} (\sqrt{\frac{1}{\alpha}-\alpha^{2}})}
    \end{cases}
    \end{equation}
	
	\vspace*{\baselineskip}
	
	\noindent
	Overall, the notch factor considers the notch taper $i$, material constant $k_{n}$, ratio of depth above notch to total depth $\alpha$, the distance to the support $x$, and the total depth of the beam h \cite{_timber_2005,jockwer_structural_2014}.
	
	\vspace*{\baselineskip} 
	
	\noindent
	\textbf{CSA O.86 Approach}\par
	
	\noindent
     The Canadian design code (CSA 0.86:2014) is also based on fracture mechanics and thus has a similar design concept to Eurocode 5 \cite{jockwer_structural_2014}. However, the design is based around the verification of the resistance of the notch ($F_{r}$) instead of the stresses, which is calculated by
     
	\begin{equation}
	F_{r} = \Phi F_{f} A K_{N} 
	\end{equation}     
     
     where\par
     $ \Phi $ = Resistance factor (given as 0.9) \par
     $ A $ = Cross-sectional area
     
     \vspace*{\baselineskip}
     
     \noindent
     $F_{f}$ can be calculated from $f_{f}$ and Equation \ref{eq:Ff} \cite{_csa_2014,_errata:_2013}.
     
   	\begin{equation}
   	F_{f} = f_{f} K_{D} K_{H} K_{Stp} K_{T}
   	\label{eq:Ff} 
   	\end{equation}  
     
     \noindent
     Where $f_{f}$ is given as 0.5MPa for sawn lumber and the condition ($K$) values can be found in CSA O.86.
 
     \vspace*{\baselineskip}
    
     \noindent    
     $K_{N}$ is the notch factor and can be found using Equation \ref{eq:kn} \cite{jockwer_structural_2014,_csa_2014}. The notch factor is based on studies by Smith and Springer, and takes into account the depth of the beam ($d$), notch ratio $\alpha$ and notch length ratio $\eta$ \cite{smith_consideration_1993}.
     
	\begin{equation}
	K_{N} = \bigg\{0.006d \bigg[1.6 \bigg(\dfrac{1}{\alpha}-1 \bigg)+\eta^{2} \bigg(\dfrac{1}{\alpha^{3}}-1\bigg) \bigg] \bigg\}^{-1/2} 
	\label{eq:kn}
	\end{equation} 
    
    where\par
    $ d $ = Total depth of member \par
    $ d_{n} $ = Depth of notch \par
    $ e $ = Length to notch from centre of support \par
    $\alpha = 1 - d_{n}/d$ \par
    $\eta = e/d$ \par
 
    \vspace*{\baselineskip}

    \noindent
    To satisfy CSA O.86 design requirements, the design shear load ($Q_{f}$) must be less than the notch resistance load ($F_{r}$) \cite{_csa_2014,_errata:_2013}. This method only considers the effects of a rectangular end notch (slope 1:0), and does not account for different notch angles.  

\subsection{Timber Strengthening}
There are many methods used to strengthen timber structures \cite{plevris_frp-reinforced_1992}. These techniques are based around combining various different forms of reinforcement to the members. Basic forms of reinforcement have been utilised such as steel bars, steel and aluminium plates and externally bonded plywood. There have also been some slightly more complex strengthening efforts. In 1965, Peterson \cite{leicester_size_1969} attempted to prestress glulam timber beams using stressed steel plates. This was achieved by fixing the steel plates to the tension side of the glulam beam using epoxy. Others include pre-stressing glulam using cable and strengthening using steel tension bearing embedded wire. 

\vspace*{\baselineskip}

\noindent
A popular form of reinforcement of timber beams is the use of fibre reinforced polymers (FRP) composites. This form of reinforcement is unique because of its ability to improve structural strength, stiffness and ductility characteristics, maintaining a very light weight \cite{plevris_frp-reinforced_1992}. Although many methods of timber strengthening has been tested, there are few recorded efforts in testing strengthened notched timber members. 


\subsubsection{Strengthening of Notched Beams}
Due to the abrupt change in the section area from a notch, high stresses are concentrated at the corner of the notch and can develop cracks. Extensions of these cracks can lead to failure and thus reinforcing notches is required to reduce the risk of cracking \cite{jockwer_structural_2014}. Notches can be reinforced either internally (Figure \ref{fig:Internal}) or externally (Figure \ref{fig:External}). Internal reinforcements are usually screwed/glued in rods and fully threaded screws. It should be noted that the screws and rods must be tight fitting to reduce the impact of shrinkage and water penetrating the hole\cite{jockwer_structural_2014,fawwaz_structural_2012}. 

	\begin{center}
		\begin{figure}[h]
			\includegraphics[scale=0.5]{Notch_Screw}
			\caption{(a) Parameters of internal reinforcement  (b) The theoretical portion of the shear stress taken by the reinforcement  \cite{jockwer_structural_2014}}
			\label{fig:Internal}
		\end{figure}
	\end{center}
	
\vspace*{\baselineskip}
	
\noindent
Bolts have been used as a standard notch strengthening method by TMR and are commonly M24 galvanised bolts, inserted perpendicular to the grain and extending through the full depth of the member. A 3mm thick plate is used at either end of the bolt to reduce cracking and pull out effects. The bolts are threaded at both ends and held on by nuts as shown in Figure \ref{fig:Internal2} \cite{_timber_2005}

	\begin{center}
		\begin{figure}[h]
			\includegraphics[scale=0.5]{Bolt_Layout}
			\caption{TMR Standard Bolt Layout \cite{_timber_2005}}
			\label{fig:Internal2}
		\end{figure}
	\end{center}
	

\noindent
External notch reinforcement methods are commonly adhered plywood, LVL or lamellas of solid timber and metal plate fasteners \cite{jockwer_structural_2014,fawwaz_structural_2012}. These are used to increase the strength capacities of the notch corner and reduce the risk of cracking. 

\vspace*{\baselineskip}

\noindent
Steel straps have also been used by TMR to reduce notching effects on piles, as seen in Figure \ref{fig:External}. However this method has not been used to reinforce girders. Essentially, a strap works similarly to a plate or wrap, where is compacts the notch corner and reduces the effects from notch cracking. 

	\begin{figure}[h]
		\includegraphics[scale=0.6]{Piles}
		\caption{Straps on Piles \cite{_timber_2005}}
		\label{fig:External}
	\end{figure} 
	\pagebreak

\noindent
Studies have been undertaken comparing reinforcement methods, however were only carried out on rectangular sectioned beams, and no other sections have been considered  \cite{jockwer_structural_2014}.

\subsection{Analysis and Modelling}
For analysis and modelling purposes, timber can be assumed to have orthotropic properties in the longitudinal, tangential and radial directions \cite{kim_modeling_2010}. These directions are shown in Figure \ref{fig:Axes} with relation to the cross-sectional grain direction, where $L$ refers to longitudinal, $R$ to radial and $T$ tangential. This will assist in developing accurate models using finite element analysis (FEA) of a three-dimensional member. 

\begin{figure}[h]
	\begin{center}
		\includegraphics[scale=0.9]{Wood_Axes}
	\end{center}
 	\caption{Wood Axes \cite{porteous_structural_2013}}
 	\label{fig:Axes}
\end{figure}

\vspace*{\baselineskip}
\pagebreak

\section{Materials and Methodology}
Notches are widely used throughout timber bridge construction as a method of seating over other members or piers. Due to the section loss caused by notching, horizontal splitting commonly occurs along the grain, which may result in failure. This is caused by high stresses occurring in the notch corners, particularly in shear, combined with a loss of member capacity due to the reduced area. Although there are design methods for notching, they differ between codes and are based on rectangular sections. This is a major issue as members commonly used in timber bridges are of circular section. To achieve a better understanding of notch behaviour and current design methods, the following experiments and numerical testing were undertaken. 

	\begin{enumerate}
		\item Small scale tests on rectangular and round members were carried out to determine the critical notch angle for each profile
	\end{enumerate}

\subsection{Preliminary Results}

\subsubsection{Finite Element Analysis}
A small scale model of the rectangular 1:0 slope notch specimen was completed on ANSYS V16.0, to established the optimum placement for data reading equipment and critical loading set-out for all tests. Figures \ref{fig:Def} to \ref{fig:fig:y_norm} show the ANSYS analysis for the rectangular end notched specimen, centrally loaded and simply supported.

\vspace*{\baselineskip}

\begin{figure}[h]
	\begin{center}
		\includegraphics[scale=0.45]{Ansys_Deflection}
	\end{center}
	
	\caption{Rectangular Section Beam: deflection}
	\label{fig:Def}
\end{figure}
\pagebreak
\begin{figure}[h]
	\begin{center}
		\includegraphics[scale=0.45]{YZ_shear_strain}
	\end{center}
	\caption{Rectangular Section Beam: XY-plane shear strain}
	\label{fig:yz_shear}
\end{figure}

\begin{figure}[h]
	\begin{center}
		\includegraphics[scale=0.45]{y_normal_strain}
		\includegraphics[scale=0.45]{z_normal_strain}
	\end{center}
	\caption{Rectangular Section Beam: normal strain in Y-direction (top) and Z-direction (bottom)}
	\label{fig:fig:y_norm}
\end{figure}

\noindent
From Figures \ref{fig:Def} to \ref{fig:fig:y_norm}, it was observed that the strains were most critical within a 5mm radius around the notch corner with the deflection being most critical at the centre of the beam. Figure \ref{fig:fig:y_norm} also indicated a large amount of compressive strain would occur directly beneath the loading plate, and tensile strain on the bottom face, in line with the load. These results ultimately defined the placement of the strain gauges and LVDT.

\pagebreak

\noindent
To establish the small scale FEA model, specific properties were used for ANSYS and can be seen in Figure \ref{fig:Properties}.
\vspace*{\baselineskip}
\begin{figure}[h]
	\begin{center}
		\includegraphics[scale=0.65]{Ansys_Properties}
	\end{center}
	\caption{Properties Used for ANSYS Simulation}
	\label{fig:Properties}
\end{figure}

\vspace*{\baselineskip}


\subsection{Experimental Set-up}
A three point loading set-up was used for all member testing throughout the experiments. The set up consisted of a centrally loaded, simply supported beam. Modelling carried out using ANSYS suggested that this location for loading would give the most critical effects and reduce any failure due to crushing over the notch. Steel plates with dimensions 5mm x 40mm x 60mm were used over the supports to distribute the load onto to beam. All timber used were of the same species and dimensions. The tests were carried out for both rectangular and circular section members and the moment of inertia and section modulus for both types of beams were assumed to approximately constant. This will allow comparison in design methods and results. A Linear Variable Differential Transformer (LVDT) was also used to measure the deflection at the centre of the beam, as well as a 600kN capacity load cell, which was was used to measure the applie load. To collect all the data from the strain gauges, LVDT and load cell, a CR3000 Campbell's scientific data logger was used.

\vspace*{\baselineskip}

\noindent
The rectangular members tested were 60mm wide x 100mm deep x 800mm long, back sawn timber. Allowing for the 30\% maximum depth loss, the notches were cut to a depth of 30mm, and were cut so that the notch corner sat 200mm in from the end of the specimen.  The specimens were notched at one end to ensure any notch failure would be concentrated and that notch. This allowed us to decrease the number of strain gauges required to record data. The test set-up for the rectangular experiments can be seen in Figure \ref{fig:rect}.

\begin{figure}[h]
	\begin{center}
		\includegraphics[scale=0.55]{Rectangular_Set_up}
	\end{center}
	\caption{Rectangular Member Test Set-Up}
	\label{fig:rect}
\end{figure}

\noindent
The supports were 600mm apart, from centre to centre, and the member was placed so it sat over the support at 100mm in from either end (100mm from notch corner). A 5mm x 40mm x 60mm steel plate was also placed at the centre of the beam and under the point load to distribute the load. The tests were set-up within a 1000 kN capacity Avery MTS machine, where the MTS applied a load at a constant rate. AS1720 specifies instantaneous load rating to be a rate at which the member fails within a time frame of 5mins. After some alteration, the load rate was set to 10kN applied load per minute to comply with AS1720. 


\noindent
The round members used were 800mm in length with a 100mm diameter, which was design to maintain a similar moment of inertia with the rectangular member. The notch was cut to a depth of 25mm, which was chosen to stay within Department of Transport and Main Road's maximum limitations of 25\% depth loss. These members were sawn so their grain profile was radial. 

\vspace*{\baselineskip}

\noindent
The same set-up was used for the round specimens as used for the rectangular; with a simply supported, 3 point load set up, 600mm effective length between supports, and the specimen supported 100mm in from either end. The same MTS machine was used at the same loading rate applied, for ease of later comparison with the rectangular specimen results. This set-up can be seen in Figure \ref{fig:round}.


\begin{figure}[h]
	\begin{center}
		\includegraphics[scale=0.55]{Circular_Setup}
	\end{center}
	\caption{Round Member Test Set-Up}
	\label{fig:round}
\end{figure}
\pagebreak

\noindent
A rounded 5mm thick by 40mm wide metal loading plate was made, curving half of the members circumference, with a 3mm x 40mm x 300mm long, straight piece of metal welded to one side to support the LVDT. This plate can be seen in Figure \ref{fig:RoundPlate}. 
\vspace*{\baselineskip}
\begin{figure}[h]
	\begin{center}
		\includegraphics[scale=0.1]{Round_Plate}
	\end{center}
	\caption{Loading and LVDT Plate for Round Specimen}
	\label{fig:RoundPlate}
\end{figure}

\noindent
\textbf{Specimen Documentation}\par
\noindent
Before any testing was undertaken, the specimens were required to be documented in great detail, to account for any inaccuracies or defect interference with experimental results. Firstly, the length, width, depth, average diameter, notch length, notch depth and mass were measured and recorded. The specimen was then carefully checked for defects and imperfections (i.e. knots, rot, cracks etc.). These were measured, photographed and documented and the process was repeated for all 24 specimens. 

\vspace*{\baselineskip}

\noindent
\textbf{Specimen Strain Gauge Implementation}\par
\noindent
The locations for the expected maximum loading effects on the beam were determined using ANSYS modelling and strain gauges were placed in these locations for all specimens tested.

\begin{figure}[h]
	\begin{center}
		\includegraphics[scale=0.4]{Gauge_Set_Up}
	\end{center}
	\caption{Strain Gauge and LVDT Layout}
	\label{fig:Gauge}
\end{figure}

\noindent
As shown in Figure \ref{fig:Gauge}, a 2mm (FLA2--11--3L) strain gauge will be placed 5mm vertically and horizontally from the notch corner with a 30mm (PFL--30--11--3L) strain gauge directly next to it. This arrangement will be mirrored on both sides of the notch, and a 30mm strain gauge will be placed centrally on the top and bottom of the centre of the beam. These strains will be used to determine an overall stress profile for the notch and the centre of the beam. This would provide valuable information on the reaction of the notches due to the applied loads

\vspace*{\baselineskip}

\noindent
The strain gauges were then attached to the specimen using the corresponding strain gauge adhesive. The 30mm strain gauges were centrally placed placed at the top and bottom of the beam, running along the grain. The 2mm and 30mm strain gauges were then placed near the notch corner as indicated in Figure \ref{fig:Gauge}. Particular care was taken to ensure that the wires were not touching or were separated with duct tape. This process was repeated for all specimens.

\vspace*{\baselineskip}

\noindent
\textbf{LVDT Implementation}\par
\noindent
For the rectangular specimens, piece of plywood (20cm x 20cm x 5cm) was attached to the specimen and mid-span using super glue. Once it was dry and secured, a 50mm LDVT was magnetically attached to the MTS machine and placed directly above the piece of plywood. For the round specimens, a long metal plate was incorporated into the constructed curved loading plate, which extruded horizontally from the centre of the beam to allow seating for the LVDT.

\vspace*{\baselineskip}

\noindent
\textbf{Camera Implementation}\par
\noindent
A 400fps camera was used to video the notch of each specimen during loading. The camera was connected to a magnetic lever arm which would magnetically attach to the MTS machine. This allowed the camera to be moved to the optimal position to capture the notch failure for each experiment.

\vspace*{\baselineskip}

\noindent
\textbf{Timber Specimen Test Set-Up}\par
\noindent
Once the strain gauges had been securely attached, the specimen was prepared to undertake loading. To set-up the experiment, the timber specimen was placed upon the simply supported arrangement within the MTS machine. This arrangement consisted of the notched end supported by a pin support and un-notched end supported by a roller, as shown in Figure \ref{fig:set_up}.

\vspace*{\baselineskip}

\begin{figure}[h]
	\begin{center}
		\includegraphics[scale=0.1]{MTS_Set-up}
	\end{center}
	\caption{Experimental Set-up in MTS Machine}
	\label{fig:set_up}
\end{figure}
\pagebreak

\noindent
The metal support plates were then slipped under the member to sit centrally above the two end supports, with extreme care taken to not interfere with any strain gauges throughout the set-up. 

\vspace*{\baselineskip}

\noindent
\textbf{Data Logger Set Up}\par
\noindent
After the specimen experimental arrangement was complete, the strain gauges, LVDT and Load Cell were connected to the data logger. This was achieved by wiring 8 ports of the data logger with long wires that had alligator clips at the opposing ends.

\vspace*{\baselineskip}

\noindent
\textbf{Timber Specimen Testing}\par
\noindent
Once the specimen was correctly placed, set-up and all strain gauges and the LVDT was connected, the load cell was placed on top of the loading plate, a steel disc and a steel ball. A 5mm x 100mm x 100mm steel loading plate was then placed on top of the load cell, and the MTS load was lowered until it was just touching the top loading plate. Before we began loading, the data logger readings from the strain gauges, LVDT and load cell were checked to determine if all were properly connected and collecting data, the overall set-up can be seen in Figure \ref{fig:SETUP}.

\vspace*{\baselineskip}

\begin{figure}[h]
	\begin{center}
		\includegraphics[scale=0.35]{SETUP}
	\end{center}
	\caption{Set-up of loading plates, metal ball and load cell in MTS Machine}
	\label{fig:SETUP}
\end{figure}

\noindent
Finally, the data logger was zeroed, and the MTS machine applied a load at a constant rate until the specimen reached ultimate failure. This set-up and testing process was the same for all 24 specimen, and all experiments were completed in the James Cook University structural engineering lab. All data from the strain gauges, LVDT and load cell was then analysed in conjunction with video footage of the experiment to align points of cracking and failure.

\subsubsection{Altering Notch Angles of Small Scale Members}
Three notch slopes were tested in this experiment; slopes 1:0, 1:2 and 1:4. These notch slopes were chosen as they are commonly used in timber bridging and have direct design methods in AS1720, which can be used for later comparison. The strains surrounding the notch and failure type were be observed to determine the critical notch angle. 

\begin{figure}[h]
	\begin{center}
		\includegraphics[scale=0.4]{Notch_Angles}
	\end{center}
	\caption{Notch Profiles for Rectangular Section Members}
	\label{fig:Rectangular2}
\end{figure}
\pagebreak

\noindent
A set of 4 specimens were tested for each notch angle in both rectangular and circular section. The notch layout for each test can be seen in Figures \ref{fig:Rectangular2} and \ref{fig:Circular}.
\vspace*{\baselineskip}
\begin{figure}[h]
	\begin{center}
		\includegraphics[scale=0.55]{Circular_Notch_Angles}
	\end{center}
	\caption{Notch Profiles for Circular Section Members}
	\label{fig:Circular}
\end{figure}

\noindent
Table 3 shows the set experimental parameters for each section shape and notch profile. As it can be seen, the moment of inertia between the two section is approximately equal, with a difference of $0.09 \times 10^{6} mm^{4}$.

\pagebreak
\captionof{table}{Experimental Parameters}

\begin{center}
	\begin{tabularx}{\textwidth}{|>{\centering}X|>{\centering}X|>{\centering}X|>{\centering}X|>{\centering}X|>{\centering}X|>{\centering}X|>{\centering}X|} 
		\hline
	    \multicolumn{8}{|c|}{\textbf{Rectangular Section}} \\
		\hline
	%	\cline{2-3}
		
		\textbf{Notch Profile} & \textbf{Notch Angle Slope} & \textbf{Member Depth (mm)} & \textbf{Width (mm)} & \textbf{Notch Depth (mm)} & \textbf{Total Section Area ($mm^{2}$)} & \textbf{Area Above Notch Corner ($mm^{2}$)} & \textbf{Moment of Inertia ($10^{3} mm^{4}$)} \tabularnewline [0.5ex] 
		\hline
		1 & 1:0 & 100 & 60 & 30 & 6000 & 4200 & 5000 \tabularnewline [0.5ex]
		\hline
		2 & 1:2 & 100 & 60 & 30 & 6000 & 4200 & 5000 \tabularnewline [0.5ex]
		\hline
		3 & 1:4 & 100 & 60 & 30 & 6000 & 4200 & 5000 \tabularnewline [0.5ex]
		\hline
	   
	    \multicolumn{8}{|c|}{\textbf{Circular Section}} \\
	    \hline
	    
	    \textbf{Notch Profile} & \textbf{Notch Angle Slope} & 
	    \multicolumn{2}{c|}{\textbf{Diameter (mm)}}
	    & \textbf{Notch Depth (mm)} & \textbf{Total Section Area ($mm^{2}$)} & \textbf{Area Over Notch Corner ($mm^{2}$)} & \textbf{Moment of Inertia ($10^{3} mm^{4}$)} \tabularnewline [0.5ex] 
	    \hline
	    1 & 1:0 & \multicolumn{2}{c|}{100} & 25 & 7850 & 6320 & 4910 \tabularnewline [0.5ex]
	    \hline
	    2 & 1:2 & \multicolumn{2}{c|}{100} & 25 & 7850 & 6320 & 4910 \tabularnewline [0.5ex]
	    \hline
	    3 & 1:4 & \multicolumn{2}{c|}{100} & 25 & 7850 & 6320 & 4910 \tabularnewline [0.5ex]
	    \hline
	\end{tabularx}
\end{center}

\vspace*{\baselineskip}

\subsection{Material Properties}
\noindent \textbf{Timber Properties}\par
\noindent
The timber used for testing was spotted gum (corymbia maculata) timber, as it a hardwood that is commonly used in Queensland bridges currently being utilised. The properties and characteristics of spotted gum timber are given in Table \ref{tab:spotty} \cite{elsener_material_2014,hopewell_spotted_2004}. All specimens were unseasoned and green (have a water content greater than 12\%) and were sourced from Grays Sawmill in Proserpine, Queensland.

\pagebreak

\captionof{table}{Properties of Spotted Gum (Corymbia Maculata) \cite{elsener_material_2014,hopewell_spotted_2004}}
\label{tab:spotty}

\begin{center}
	\begin{tabularx}{\textwidth}{|>{\centering}X|>{\centering}X|>{\centering}X|>{\centering}X|>{\centering}X|}	
		\hline 
		
		\multicolumn{2}{|c|}{\textbf{Properties}} & \textbf{Moisture Content 12\%} & \textbf{Green (MC $\textgreater$ 12\%)}  & \textbf{Dry (MC $\textless$ 12\%)} \tabularnewline  [0.5ex]
		\hline
		
		\multirow{3}{*}{\parbox{2.5cm}{\centering Modulus of Rupture MOR (MPa)}}& \textit{Longitudinal} & 141.1 & 99 & 150 \tabularnewline  [0.5ex] 
		\cline{2-5}
		& \textit{Radial} & 19.5 &  &  \tabularnewline [0.5ex] 
		\cline{2-5}
		& \textit{Tangential} & 14.9 &  &  \tabularnewline [0.5ex] 
		\hline
		
		\multirow{3}{*}{\parbox{2.5cm}{\centering Modulus of Elasticity MOE (MPa)}} & \textit{Longitudinal} & 26174 & 18000 & 23000 \tabularnewline [0.5ex] 
		\cline{2-5}
		& \textit{Radial} & 2405 & 1531 &  \tabularnewline [0.5ex] 
		\cline{2-5}
		& \textit{Tangential} & 1499 & 665 &  \tabularnewline [0.5ex] 
		\hline
		
		\multirow{3}{*}{\parbox{2.5cm}{\centering Shear Modulus G (MPa)}}& \textit{Long - Rad} & 1736 &  &  \tabularnewline [0.5ex] 
		\cline{2-5}
		& \textit{Rad - Tang} & 840 &  &  \tabularnewline [0.5ex] 
		\cline{2-5}
		& \textit{Long - Tang} & 1530 &  &  \tabularnewline [0.5ex] 
		\hline
		
		\multirow{6}{*}{\parbox{2.5cm}{\centering Poisson's Ratio $\nu$}}& \textit{Long - Rad} & 0.49 &  &  \tabularnewline [0.5ex] 
		\cline{2-5}
		& \textit{Long - Tang} & 0.550 &  &  \tabularnewline [0.5ex] 
		\cline{2-5}
		& \textit{Rad - Tang} & 0.660 &  &  \tabularnewline [0.5ex] 
		\cline{2-5}
		& \textit{Rad - Long} & 0.045 &  &  \tabularnewline [0.5ex] 
		\cline{2-5}
		& \textit{Tang - Rad} & 0.480 &  &  \tabularnewline [0.5ex] 
		\cline{2-5}
		& \textit{Tang - Long} & 0.047 &  &  \tabularnewline [0.5ex] 
		\hline
		
		
		\multicolumn{2}{|c|}{Bending Strength (MPa)}& 142 &  &   \tabularnewline [0.5ex] 
		\hline
		
		\multicolumn{2}{|c|}{Compressive Strength (MPa)}& 76 &  &   \tabularnewline [0.5ex] 
		\hline
		
		\multicolumn{2}{|c|}{Tensile Strength (MPa)}& 159 &  &   \tabularnewline [0.5ex] 
		\hline
		
		\multicolumn{2}{|c|}{Density ($kg/m^{3}$)}& 1060 & 1150 & 1100  \tabularnewline [0.5ex] 
		\hline
		
		\multicolumn{2}{|c|}{Strength Group}&  & S2 & SD2  \tabularnewline [0.5ex] 
		\hline
		
		\multicolumn{2}{|c|}{F-Grade}&  & F14 & F22  \tabularnewline [0.5ex] 
		\hline
		
	\end{tabularx}
\end{center}

\section{Results and Discussion}

\subsection{Rectangular Specimens}
\subsubsection{Loading Rates}
The loading rates used to load the rectangular specimens can be seen in Figure \ref{fig:Rect_load}. It can be seen that the load rate was constant throughout loading all specimens, at a load rate of 10kN per minute. 

\begin{figure}[h]
	\begin{center}
		\fbox{\includegraphics[scale=0.7]{Rect_load}}
	\end{center}
	\caption{Load rates for rectangular specimens}
	\label{fig:Rect_load}
\end{figure}

\noindent
It can also be seen from Figure \ref{fig:Rect_load} that every specimen experienced an extreme drop in load at some point throughout the experiment. The reason for this was at first presumed to be due to the notch crack initiating at this point. However, when compared to the actual loads at which the notch crack initiates, see Table \ref{tab:rect_Capac}, the drop appears to take place at a higher load. Therefore, it is presumed that the extreme decrease in load is the point at which the notch crack fully propagates passed the centre of the beam. This theory seems plausible as when the crack fully propagates, the member is essentially changing section, as the only part of the member taking the load, would be the remaining section above the crack, thus the reason for the sudden drop in load.  

\subsubsection{Failure Modes}
Table \ref{tab:Rect_Fail_Type} shows the water content and initial failure type of each specimen. From this is can be seen that the notch corner cracked on every specimen before ultimate failure occurred.

\vspace*{\baselineskip}

\captionof{table}{Rectangular Specimen Water Contents and Initial Failure Locations}
\begin{center}
	\begin{tabular}{|c|c|c|c|} 
		\hline
		
		\textbf{Notch Angle Slope} & \textbf{Specimen No.} & \textbf{Water Content (\%)} & \textbf{Location of Initial Failure}\\ [0.5ex]
		\hline
		
		1:0 & 1 &  & Cracked at notch corner \\ [0.5ex]
		\hline
		1:0 & 2 &  & Cracked at notch corner \\ [0.5ex]
		\hline
		1:0 & 3 &  & Cracked at notch corner \\ [0.5ex]
		\hline
		1:0 & 4 & 19 & Cracked at notch corner \\ [0.5ex]
		\hline
		
		1:2 & 1 & 14 & Cracked at notch corner \\ [0.5ex]
		\hline
		1:2 & 2 & 16 & Cracked at notch corner \\ [0.5ex]
		\hline
		1:2 & 3 & 16 & Cracked at notch corner \\ [0.5ex]
		\hline
		1:2 & 4 & 13 & Cracked at notch corner \\ [0.5ex]
		\hline
		
		1:4 & 1 & 19 & Cracked at notch corner \\ [0.5ex]
		\hline
		1:4 & 2 & 15 & Cracked at notch corner \\ [0.5ex]
		\hline
		1:4 & 3 & 18 & Cracked at notch corner \\ [0.5ex]
		\hline
		1:4 & 4 & 18 & Cracked at notch corner \\ [0.5ex]
		\hline
	\end{tabular}
	\label{tab:Rect_Fail_Type}
\end{center}

\vspace*{\baselineskip}
\noindent
It should be noted that the water contents for rectangular specimens 1, 2 and 3 of 1:0 notch were unable to be taken due to lack of required equipment.

\vspace*{\baselineskip}

\noindent
Throughout all rectangular experiments, there was a definite pattern in the types of failure that occurred. The first sign of failure occurred at the notch corner, where a crack would appear. The crack would initiate by first opening perpendicular to the grain at the notch corner (notch failure mode 1), due to excessive tensile forces, which can be seen in Figure \ref{fig:Rect_Crack}. 

\vspace*{\baselineskip}

\begin{figure}[h]
	\begin{center}
		\includegraphics[scale=0.5]{Rect_Crack_Open}
	\end{center}
	\caption{Specimen 4 notch 1:0, notch crack opening}
	\label{fig:Rect_Crack}
\end{figure}
\pagebreak

\noindent
The crack would then shear through the member (notch failure mode 2) and propagate to a point just beyond the centre of the beam, as seen in Figure \ref{fig:Rect_Prop}. The notch corner crack then opened (failure mode 3) due to a combination of tensile and shear forces, which began after the crack initially sheared. This failure at the notch corner can also be clearly seen in Figure \ref{fig:Rect_Prop}.  

\vspace*{\baselineskip}

\begin{figure}[h]
	\begin{center}
		\includegraphics[scale=0.31]{Rect_propegate}
	\end{center}
	\caption{Specimen 4 notch 1:0, notch crack shearing and opening}
	\label{fig:Rect_Prop}
\end{figure}
\pagebreak

\noindent
After the notch crack had fully propagated passed the centre of the beam, the specimens ultimately failed in shear as can been seen in Figure \ref{fig:Rect_Shear}. 

\vspace*{\baselineskip}

\begin{figure}[h]
	\begin{center}
		\includegraphics[scale=0.31]{Rect_Shear}
	\end{center}
	\caption{Specimen 4 notch 1:0, Shear Failure}
	\label{fig:Rect_Shear}
\end{figure}

\noindent
Almost immediately after ultimate shear failure occurred, the specimens then failed in flexure at the centre of the beam, as shown in Figure \ref{fig:Rect_Flex}. The overall succession of notch failure and ultimate failure occurred as expected; sequencing through the notch failure modes and ultimately failing in shear or flexure. 

\vspace*{\baselineskip}

\begin{figure}[h]
	\begin{center}
		\includegraphics[scale=0.3]{Rect_Flexure}
	\end{center}
	\caption{Specimen 4 notch 1:0, Flexural Failure}
	\label{fig:Rect_Flex}
\end{figure}
\pagebreak

\noindent
All rectangular specimens ultimately failed in shear and then flexure, apart from specimen 2 of the 1:0 slope notch, which only failed in flexure, as shown in Figure \ref{fig:Spec_2}.

\begin{figure}[h]
	\begin{center}
		\includegraphics[scale=0.05]{Spec_2_flex}
		\includegraphics[scale=0.086]{Spec_2_Top}
	\end{center}
	\caption{Specimen 2 notch 1:0 failure; side view left, top view right}
	\label{fig:Spec_2}
\end{figure}

\noindent
As there was no obvious defects present on the specimen, we can only speculate on the reason why it did not fail in shear. A possible reason could be that the specimen may have had a very high water content at the centre of the beam significantly decreasing its flexural capacity in this region. This theory may be supported in Figure \ref{fig:Spec_2_wet} where the specimen appears to be very wet in the specified area. However, as the water content was not taken for this specific specimen, this theory can not be verified. 

\begin{figure}[h]
	\begin{center}
		\includegraphics[scale=0.1]{Spec_2_wet}
	\end{center}
	\caption{Specimen 2 notch 1:0; inside centre of beam}
	\label{fig:Spec_2_wet}
\end{figure}

\noindent
The ultimate shear failure occurred approximately through the middle of the cross-section of one half of the beam. However the location alternated between, over the notch half and over the roller support half (opposing end to the notch), throughout the experiments. A summary of these locations can be seen in Table \ref{tab:Rect_Shear_Fail}.

\vspace*{\baselineskip}

\captionof{table}{Rectangular Specimen Shear Failure Locations}
\begin{center}
	\begin{tabularx}{\textwidth}{|>{\centering}X|>{\centering}X|>{\centering}X|} 
		\hline
		
		\textbf{Notch Angle Slope} & \textbf{Specimen No.} & \textbf{Shear Failure Location}\tabularnewline [0.5ex]
		\hline
		
		1:0 & 1 & Through section over roller support \tabularnewline [0.5ex]
		\hline
		1:0 & 2 & N/A \tabularnewline [0.5ex]
		\hline
		1:0 & 3 & Through section over roller support \tabularnewline [0.5ex]
		\hline
		1:0 & 4 & Through section over the notch \tabularnewline [0.5ex]
		\hline
		
		1:2 & 1 & Through section over the notch \tabularnewline [0.5ex]
		\hline
		1:2 & 2 & Through section over the notch \tabularnewline [0.5ex]
		\hline
		1:2 & 3 & Through section over the notch \tabularnewline [0.5ex]
		\hline
		1:2 & 4 & Through section over roller support \tabularnewline [0.5ex]
		\hline
		
		1:4 & 1 & Through section over the notch \tabularnewline [0.5ex]
		\hline
		1:4 & 2 & Through section over the notch \tabularnewline [0.5ex]
		\hline
		1:4 & 3 & Through section over the notch \tabularnewline [0.5ex]
		\hline
		1:4 & 4 & Through section over roller support \tabularnewline [0.5ex]
		\hline
	\end{tabularx}
	\label{tab:Rect_Shear_Fail}
\end{center}

\vspace*{\baselineskip}

\noindent
A possible reason for the differing shear failures could be that, once the notch crack had sheared passed the centre of the beam and opened, the cross-section of the entire beam was reduced to that over the notch. Hence the shear area and capacity became the same throughout the member, increasing the probability that it would fail at any point on the beam and differ due to each member's particular grain arrangement.

\subsubsection{Specimen Capacities}
The overall capacities and times taken to for the failure to occur, for all rectangular specimens, are shown in Table \ref{tab:rect_Capac}. These results were determined through analysis of the load cell, LVDT and strain gauge data as well as through visual correlation using the footage taken of each specimen's notch and ultimate failures.

\vspace*{\baselineskip}

\captionof{table}{Rectangular Specimen Crack Inititation and Ultimate Loads}
\begin{center}
	\begin{tabularx}{\textwidth}{|>{\centering}X|>{\centering}X|>{\centering}X|>{\centering}X|>{\centering}X|>{\centering}X|} 
		\hline
		
		\textbf{Notch Angle Slope} & \textbf{Specimen No.} & \textbf{Notch Crack Initiation (kN)} & \textbf{Time for Notch to Crack (min:sec)} & \textbf{Ultimate Failure (kN)} & \textbf{Time to Ultimate Failure (min:sec)} \tabularnewline [0.5ex] 
		\hline
		1:0 & 1 & 30.11 & 02:58 & 47.16 & 04:41 \tabularnewline [0.5ex]
		\hline
		1:0 & 2 & 23.88 & 02:22 & 34.39 & 03:29 \tabularnewline [0.5ex]
		\hline
		1:0 & 3 & 16.08 & 01:33 & 27.97 & 02:46 \tabularnewline [0.5ex]
		\hline
		1:0 & 4 & 23.27 & 02:18 & 41.94 & 04:12 \tabularnewline [0.5ex]
		\hline
		
		1:2 & 1 & 26.89 & 02:39 & 42.56 & 04:14 \tabularnewline [0.5ex]
		\hline
		1:2 & 2 & 21.53 & 02:03 & 39.77 & 03:57 \tabularnewline [0.5ex]
		\hline
		1:2 & 3 & 23.65 & 02:20 & 33.40 & 03:23 \tabularnewline [0.5ex]
		\hline
		1:2 & 4 & 29.24 & 02:52 & 40.20 & 04:22 \tabularnewline [0.5ex]
		\hline
		
		1:4 & 1 & 38.46 & 03:52 & 41.79 & 04:14 \tabularnewline [0.5ex]
		\hline
		1:4 & 2 & 49.23 & 04:51 & 49.23 & 04:51 \tabularnewline [0.5ex]
		\hline
		1:4 & 3 & 36.80 & 03:35 & 39.04 & 03:49 \tabularnewline [0.5ex]
		\hline
		1:4 & 4 & 29.84 & 02:52 & 34.51 & 03:20 \tabularnewline [0.5ex]
		\hline
		
	\end{tabularx}
	\label{tab:rect_Capac}
\end{center}

\vspace*{\baselineskip}

\noindent
From Table \ref{tab:rect_Capac}, it can be seen that the results for both the crack initiation load (notch capacity) and the ultimate capacity vary substantially throughout the specimens for each notch slope. This is most likely due to a combination of differing water contents within the specimens and any existing defects. When the results from this table are compared with the water contents, see Table \ref{tab:Rect_Fail_Type}, it can be observed for most specimens that the those with a higher water content yield a lower ultimate capacity. This pattern is relevant to all specimens for notch slopes 1:2 and 1:4 except for specimen 1 of 1:4 notch, which yielded a slightly higher capacity than those of a lower water content. This outlier may be due to specimens 3 and 4 having defects, thus reducing their capacity. A defect was observed for specimen 4 but no surface defect was observed for specimen 3, see Appendix A. Another reason for this may be due to an inaccurate water content reading of one of the specimens, or the differing grain arrangement within the specimens. For the 1:0 slope specimens, a pattern with the capacity and water content was unable to be established as the water content was not taken for the first three specimens. However, when observing the specimen characteristics, given in Appendix A, it can be observed all specimens had different defects, thus would be expected to yield various capacities. As for the notch capacities, the same trend was apparent, where the higher the water content, the lower the capacity. Again, the capacities differed due to existing defects and their location, for example specimen 3 of 1:0 slope yielded particular low capacities, having a 15mm deep crack on top of the specimen, over the notch that spanned to the centre of the beam. The same discrepancy occurred with specimen 1 of 1:4 slope, where it yielded a higher notch capacity than those with a lower water content. This is indicative that the water content for this specimen may be inaccurate.  

\vspace*{\baselineskip}

\noindent
A graph summarising the information given in Table \ref{tab:rect_Capac} is shown in Figure \ref{fig:Rect_Spec_Cap}, where the solid bars indicate the average capacity and the error bars show the variance in the results. From this graph, there is a strong trend showing that both the notch capacity and ultimate capacity increase with increasing notch slope. The ultimate capacity experiences a gradual but significant average increase of 1kN between 1:0 and 1:2 slope and 2kN between 1:2 and 1:4 slope.  

\vspace*{\baselineskip}

\begin{figure}[h]
	\begin{center}
		\fbox{\includegraphics[scale=0.65]{Rect_Spec_Cap}}
	\end{center}
	\caption{Capacities for Rectangular Section Specimens}
	\label{fig:Rect_Spec_Cap}
\end{figure}

\noindent
The same trend is observed for the notch capacities, however there is a more dramatic increase between the average notch capacity of a 1:2 slope and a 1:4 slope; where the difference between a 1:0 and 1:2 slope notch capacity is 2kN and between a 1:2 and 1:4 slope is 13kN. There was a large amount of variance in the results, however this large increase between these notch capacities would support AS1720 and TMR's recommendation to use a 1:4 slope as it appears to significantly improve the notch capacity.

\vspace*{\baselineskip}

\noindent
The overall trend that can be observed in Figure \ref{tab:rect_Capac} supports the common theory that a notch which is chamfered will have a greater capacity than a rectangular end notch (slope 1:0), and thus the critical notch angle for a rectangular section beam is a 1:0 slope.

\vspace*{\baselineskip}

\noindent
Another relationship taken from the results in Table \ref{tab:rect_Capac} can be seen in Figure \ref{fig:Rect_Time}, which shows the time between the notch crack initiating and ultimate failure. The bars in Figure \ref{fig:Rect_Time} represent the average time interval for each notch slope and the error bars shows the variance in the results.  

\vspace*{\baselineskip}

\begin{figure}[h]
	\begin{center}
		\fbox{\includegraphics[scale=0.8]{Rect_Time}}
	\end{center}
	\caption{Time between notch initiating and ultimate failure for rectangular section specimens}
	\label{fig:Rect_Time}
\end{figure}

\noindent
From this graph, it can be seen that there is an increase of 1 second between the 1:0 slope and 1:2 slope notch. This slight increase in time may be indicative of an average pattern or may be due to the average water content in the 1:2 slope notch being greater than those in the 1:0 slope. Unfortunately this theory can be neither denied or confirmed due to the unknown water contents for the 1:0 slopes, thus the reason for this increase in time interval is ultimately unknown. The most interesting trend observed from Figure \ref{fig:Rect_Time} is that time average interval for the 1:4 slope notch is over a minute less than for the 1:0 slope and 1:2 slope, with an average of 16sec. This is not unexpected as it can be seen from Figure \ref{fig:Rect_Spec_Cap}, that by the time the notch crack initiates, the specimen is almost at ultimate capacity. However, when studying the results shown in Figure \ref{fig:Rect_Time}, it becomes apparent that using a 1:4 chamfer notch may lead to notch failure occurring very close to or at the point of ultimate failure. From a maintenance perspective, this could be a major issue, as signs of notch cracking are hard to identify and are usually considered as means to observe the defect on a one to six monthly basis. As this experiment was conducted under short-term loading, it is hard to specify the time interval between crack initiation at the notch and ultimate failure for a long-term load. Nevertheless research into the effects of long-term loading on the time between crack initiation at the notch corner and ultimate failure is warranted. This will determine if notching cracking is a sign of ultimate failure for 1:4 notch slope, and possibly alter current maintenance schemes for timber bridges.


\subsubsection{Strain Gauge Analysis}
All strain gauge data shown in the following graphs start from an applied load of zero and end at the point of ultimate failure or when the strain gauge cut out, whichever came first. It should also be noted than the additional weight from loading plates and loading cell has been taken into account. A strain gauge analysis is carried out below for each of the three differing notch slopes.

\vspace*{\baselineskip}

\noindent
\textbf{Notch Slope 1:0}\par
\noindent
The load-displacement graph from the LVDT results, for the rectangular section specimens, with a notch slope of 1:0 is shown in Figure \ref{fig:Rect_10_def}. The trend in most of the specimens load vs deflection does not follow usual trends and is unexpected.
\vspace*{\baselineskip}
\begin{figure}[h]
	\begin{center}
		\fbox{\includegraphics[scale=0.8]{Rect_10_Def}}
	\end{center}
	\caption{Load-Displacement Graph for Rectangular Beams with Notches of Slope 1:0}
	\label{fig:Rect_10_def}
\end{figure}

\noindent
There does not appear to be any distinct trend in the load-displacement graph for specimen 1. This is assumed to be caused by the LVDT results being inaccurate from the plywood supporting the LVDT being too weak. As for specimens 2, the load-displacement graph follows the usual pattern, then ultimately changes direction at the point at which the notch crack initiates. The same pattern could be said for specimen 3, however it experiences more of a sudden drop which occurs after the notch crack initiates. This particular discrepancy is similar to that experienced in the loading chart, presumably due to the notch crack fully propagating through the specimen, and may be suggestive of this. The load-displacement graph for specimen 4 uncommon in nature for no known reason.   
\vspace*{\baselineskip}

\begin{figure}[h]
	\begin{center}
		\fbox{\includegraphics[scale=0.75]{Rect_10_Centre}}
	\end{center}
	\caption{Top and Bottom Strain at Centre for Rectangular Beams with Notches of Slope 1:0}
	\label{fig:Rect_10_centre}
\end{figure}

\pagebreak

\noindent
The strains at the top (compressive) and bottom (tensile) of each specimen throughout loading are shown in Figure \ref{fig:Rect_10_centre}. It can be seen for specimens 1, 2 and 3, that they follow the same trend where both tension and compression strains increase linearly to a spike in the data at the same load. After this, they both then experience a significant drop in load. This spike and sudden drop in load is suggestive of a sudden change in capacity and is assumed to be the point at which the notch crack fully propagates through the beam. Specimen 4 essentially followed the same trend, however there was no significant spike in the data, and instead experienced a smooth curve. This may be due to the notch crack in this particular specimen propagating more gradually than the other specimens. After the turning point for all specimens, the tensile strain gradually decreased to zero and the compressive strains continued to a strain of roughly --8000 $\times 10^{-6}$ where it suddenly stopped increasing in strain and only increased in load until ultimate failure.
  
\vspace*{\baselineskip}

\begin{figure}[h]
	\begin{center}
		\fbox{\includegraphics[scale=0.8]{Rect_10_Z}}
	\end{center}
	\caption{Horizontal Strain for Rectangular Beams with Notches of Slope 1:0}
	\label{fig:Rect_10_Z}
\end{figure}
\pagebreak

\noindent
The average horizontal strain at the notch corner for notches of slope 1:0 can be seen in Figure \ref{fig:Rect_10_Z}, where the horizontal strains were collected from either side of the notch and averaged across the section. From the graph, it can be seen that specimens 1, 2 and 3 experience similar trends in horizontal strain, where they increase and then suddenly drop at the same load at which the compressive and tensile strains dropped. The results obtained for specimen 4are erratic and do not follow the same trend. However, this behaviour may be indicative of a slow growing notch crack.

\vspace*{\baselineskip}

\noindent
The vertical strains were also averaged across both sides of the notch, as shown in Figure \ref{fig:Rect_10_Y}. From this graph, a pattern can be seen for specimens 1 and 3, where there is a large spike in the Y strain at a load, after which the strain rapidly drops to below zero. The load at which this data drops aligns with the load at which the notch crack initiates. The results for specimen 4 are erratic and a definite trend is unable to be established. 

\vspace*{\baselineskip}

\begin{figure}[h]
	\begin{center}
		\fbox{\includegraphics[scale=0.8]{Rect_10_Y}}
	\end{center}
	\caption{Vertical Strain for Rectangular Beams with Notches of Slope 1:0}
	\label{fig:Rect_10_Y}
\end{figure}
\pagebreak

\noindent
The data for specimen 1 appears to be constantly increasing until ultimate failure. There is a small discrepancy in the data around the load at which the notch corner crack initiated, however the point at which this occurs is not clearly defined. However, the results obtained for the vertical strain in specimen 2 may be inaccurate as it does not follow the trend observed in the other specimens. This may be inaccurate due to a large wedge cut out of the notched end of the beam, as shown in Figure \ref{fig:Rect_2_wedge}, which may have been in line with the strain gauge and obscuring the data. 
\vspace*{\baselineskip}

\begin{figure}[h]
	\begin{center}
		\includegraphics[scale=0.065]{End_wedge}
		\includegraphics[scale=0.065]{Top_wedge}
	\end{center}
	\caption{Rectangular specimen 2 notch slope 1:0; wedge defect}
	\label{fig:Rect_2_wedge}
\end{figure}

\pagebreak

\noindent
\textbf{Notch Slope 1:2}\par
\noindent
The load-displacement graph for the 1:2 notch specimens can be seen in Figure \ref{fig:Rect_12_def}. Specimens 1, 3 and 4 follow the same distinct pattern distinguished in the 1:0 slope load-displacement graph, where the load increases until it suddenly drops at the point at which the notch crack is assumed to have fully propagated. After this point the deflection for specimens 3 and 4 start to decrease while the deflection for specimen 1 continues to increase. This behaviour may be due to the different grain arrangement and how each particular notch crack grows through the grain. However, specimen 2 follows the standard load-displacement relationship and does not experience a distinct discrepancy. Instead, it curves at the crack propagation load, then continues to increase in an almost linear manner. This follows a similar pattern as specimen 4 of the 1:0 slope notch, which was presumed to be due to the notch crack propagating at a more gradual rate instead of a sudden fracture. 

\vspace*{\baselineskip}

\begin{figure}[h]
	\begin{center}
		\fbox{\includegraphics[scale=0.8]{Rect_12_Def}}
	\end{center}
	\caption{Load-Displacement Graph for Rectangular Beams with Notches of Slope 1:2}
	\label{fig:Rect_12_def}
\end{figure}

\noindent
Figure \ref{fig:Rect_12_centre} shows the top and bottom strains taken from the centre of the beam for the 1:2 notch slope specimens. 

\vspace*{\baselineskip}

\begin{figure}[h]
	\begin{center}
		\fbox{\includegraphics[scale=0.75]{Rect_12_Centre}}
	\end{center}
	\caption{Top and Bottom Strain at Centre for Rectangular Beams with Notches of Slope 1:2}
	\label{fig:Rect_12_centre}
\end{figure}
\pagebreak

\noindent
The same trends were can be observed for the strains in Figure \ref{fig:Rect_12_centre} as were for the 1:0 slope tensile and compressive strains; where both strains spiked at the point at which the notch crack fully propagates. The only difference between the trends for the 1:2 slope compared to the 1:0 slope, is the compressive strains did not hit a point where the strain stopped increasing and only the load increased. Specimens 1, 3 and 4 followed this specified trend, where specimen 2 followed the same trend as specimen 4 for the 1:0 slope, again presumptively due to the crack growth being less sudden than the other specimens.

\vspace*{\baselineskip}
\noindent
The average horizontal strain data can be seen in Figure \ref{fig:Rect_12_Z}, where specimens 1, 3 and 4 experience the same trend as for the 1:0 slope specimens, increasing linearly and experiencing a sudden drop at the point the notch crack fully propagates. However, after this point the trend slightly differs, as all three specimens strain continue to increase at roughly the same rate. 

\vspace*{\baselineskip}

\begin{figure}[h]
	\begin{center}
		\fbox{\includegraphics[scale=0.75]{Rect_12_Z}}
	\end{center}
	\caption{Horizontal Strain for Rectangular Beams with Notches of Slope 1:2}
	\label{fig:Rect_12_Z}
\end{figure}
\pagebreak

\noindent
The strain underwent by specimen 2 followed a linear trend to the point at which the notch crack initiated, but failed to display the sudden drop in load observed in the other three specimens. This is most likely due to the large cut into the end of the specimen over the notch, see Figure \ref{fig:Rect_Cut}, which would have interfered with the strain gauge data. 

\vspace*{\baselineskip}

\begin{figure}[h]
	\begin{center}
		\includegraphics[scale=0.065]{Cut_Top}
		\includegraphics[scale=0.065]{Cut_Side}
	\end{center}
	\caption{Rectangular specimen 2 notch slope 1:2; thick cut defect}
	\label{fig:Rect_Cut}
\end{figure}
\pagebreak

\noindent
The average vertical strain for the 1:2 slope specimens are shown in Figure \ref{fig:Rect_12_Y} and follow the same pattern as observed for the 1:0 slope vertical strains, where the strains would increase to the crack initiation point, then severely decrease. This trend is relevant through specimens 1, 3 and 4, with the exception of specimen 2, whuih may have behaved differently due to the defect present.
\vspace*{\baselineskip}

\begin{figure}[h]
	\begin{center}
		\fbox{\includegraphics[scale=0.8]{Rect_12_Y}}
	\end{center}
	\caption{Vertical Strain for Rectangular Beams with Notches of Slope 1:2}
	\label{fig:Rect_12_Y}
\end{figure}

\vspace*{\baselineskip}

\noindent
\textbf{Notch Slope 1:4}\par
\noindent
The load-displacement graph for the 1:4 can be seen in Figure \ref{fig:Rect_14_def}. Specimen 4 follows the same trend that was observed throughout the previous experiments, where the graph follows an expected load-deflection relationship until the point at which the notch crack propagates, where it then drops and the trend changes. The same expected load-deflection relationship can be seen for specimens 2 and 3, however neither experience a sudden drop in load. This is because for specimen 2, the notch crack initiated at the point of at ultimate failure, thus the trend didn't curve or experience a extreme drop. As for Specimen 3, the notch crack initiated very close to ultimate failure, thus the trend curved but there was no drop as the crack did not fully propagate before ultimate failure. Whereas specimen 1 experienced a less defined trend. The reason for this is unknown as there was no observed defects on the specimen before testing.

\vspace*{\baselineskip}

\begin{figure}[h]
	\begin{center}
		\fbox{\includegraphics[scale=0.8]{Rect_14_Def}}
	\end{center}
	\caption{Load-Displacement Graph for Rectangular Beams with Notches of Slope 1:4}
	\label{fig:Rect_14_def}
\end{figure}

\noindent
The same trends for the top and bottom strains at the centre of the beam were experienced for the 1:4 slope specimens as both the 1:2 and 1:0 slope specimens, see Figure \ref{fig:Rect_14_centre}. Except, specimens 2 and 3 did not experience a sudden drop in strain as the notch crack did not fully propagate before ultimate failure. 

\begin{figure}[h]
	\begin{center}
		\fbox{\includegraphics[scale=0.75]{Rect_14_Centre}}
	\end{center}
	\caption{Top and Bottom Strain at Centre for Rectangular Beams with Notches of Slope 1:4}
	\label{fig:Rect_14_centre}
\end{figure}

\pagebreak

\noindent
The average horizontal and vertical strains experienced by the 1:4 notch specimens can be seen in Figure \ref{fig:Rect_14_Z_Y}. Both these strains overall follow the same trend throughout the specimens as distinguished for the 1:0 and 1:2 slope specimens; where specimen 2 shows no discrepancies as the notch doesn't crack before failure and specimen 3 shows a curved decrease in vertical strain at crack initiation. However, although specimen 1 follows a similar trend, after the notch initiates, both the horizontal and vertical strains become erratic. This is indicative that a defect may have been present within the specimen.

\pagebreak

\begin{figure}[h]
	\begin{center}
		\fbox{\includegraphics[scale=0.69]{Rect_14_Z}}
		\fbox{\includegraphics[scale=0.69]{Rect_14_Y}}
	\end{center}
	\caption{Horizontal Strain (Top) and Vertical Strain (Bottom) for Rectangular Beams with Notches of Slope 1:4}
	\label{fig:Rect_14_Z_Y}
\end{figure}

\subsection{Round Specimens}
\subsubsection{Loading Rates}
The loading rates used for the round specimens can be seen in Figure \ref{fig:Round_load}. The load rate was constantly 10kN per minute for all specimens, except for specimens 1 and 3 for notch slopes 1:0, which experienced a load rate of 8kN per minute. This error in loading rate was due to an error in the MTS loading settings which was fixed for the rest of the experiments.

\vspace*{\baselineskip}

\begin{figure}[h]
	\begin{center}
		\fbox{\includegraphics[scale=0.7]{Round_load}}
	\end{center}
	\caption{Load rates for round specimens}
	\label{fig:Round_load}
\end{figure}

\noindent
The same extreme drop in load occurred in most of the round specimens as did in the rectangular specimens, however were less prominent (i.e. less of a decrease) than those present in Figure \ref{fig:Rect_load}. Another difference between the round and rectangular specimen loading graphs were that some of the round specimens did not experience an extreme decrease in load. As it was presumed for the rectangular specimens that the decrease was due the notch fully propagating, it would be safe to assume the same for the round specimens, where the ones that didn't experience a drop, did not crack at the notch corner. 

\vspace*{\baselineskip}

\noindent
It should also be noted that specimen 1 of notch slope 1:2 was unable to be plotted on this graph as the load cell malfunctioned during the loading and the load was then taken from the MTS. The load data collected from the MTS machine was taken at slightly different load points than the data taken using the data logger, thus the loads and times were unable to be exactly aligned. However, the MTS was set to load this specimen at a rate of 10kN per minute.  

\subsubsection{Failure Modes}
Table \ref{Round_Water} shows the water contents of the round specimens and their initial modes of failure. As it can be seen, not all specimens initially failed at the notch corner. Specimens 1 of 1:0 slope and 2 of 1:4 slope initially failed at a different location due to existing defects. Whereas specimen 4 of 1:4 slope did not fail at the notch corner until ultimate failure occurred. 

\vspace*{\baselineskip}

\captionof{table}{Round Specimen Water Contents and Initial Failure Locations}

	\begin{tabular}{|c|c|c|p{4.5cm}|} 
		\hline
		\textbf{Notch Angle Slope} & \textbf{Specimen No.} & \textbf{Water Content (\%)} & \textbf{Location of Initial Failure}\\ [0.5ex]
		\hline
		
		1:0 & 1 & 22 & Cracked propagated from existing crack at opposite end, then cracked at notch corner  \\ [0.5ex]
		\hline
		1:0 & 2 & 21 & Cracked at notch corner  \\ [0.5ex]
		\hline
		1:0 & 3 & 24 & Cracked at notch corner  \\ [0.5ex]
		\hline
		1:0 & 4 & 26 & Cracked at notch corner \\ [0.5ex]
		\hline
		
		1:2 & 1 & 23 & Cracked at notch corner \\ [0.5ex]
		\hline
		1:2 & 2 & 30 & Cracked at notch corner \\ [0.5ex]
		\hline
		1:2 & 3 & 22 & Cracked at notch corner \\ [0.5ex]
		\hline
		1:2 & 4 & 29 & Cracked at notch corner \\ [0.5ex]
		\hline
		
		1:4 & 1 & 24 & Cracked at notch corner \\ [0.5ex]
		\hline
		1:4 & 2 & 22 & Cracked propagated half way down chamfer, then cracked at notch corner \\ [0.5ex]
		\hline
		1:4 & 3 & 27 & Cracked at notch corner \\ [0.5ex]
		\hline
		1:4 & 4 & 23 & Notch cracked at ultimate failure \\ [0.5ex]
		\hline
		
	\end{tabular}
	\label{Round_Water}

\vspace*{\baselineskip}

\noindent
The exact failure mode for the round specimen is not as clear as it was for the rectangular specimens. As can be seen in Table \ref{tab:Round_Fail}, the round specimens generally failed in flexure at the centre of the beam, however that were some that failed in shear. The reason for this is presumed to be due to defects effecting the shear capacity. It was noted that a large number of defects were present on the round specimens before testing, see Appendix B, and it is highly likely that defects were present within the specimens that were not accounted for, such as internal rot. 

\captionof{table}{Round Specimen Ultimate Failure Modes}
\begin{center}
	\begin{tabularx}{\textwidth}{|>{\centering}X|>{\centering}X|>{\centering}X|} 
		\hline
		
		\textbf{Notch Angle Slope} & \textbf{Specimen No.} & \textbf{Shear Failure Location}\tabularnewline [0.5ex]
		\hline
		
		1:0 & 1 & Shear failure due to existing defect \tabularnewline [0.5ex]
		\hline
		1:0 & 2 & Flexural failure \tabularnewline [0.5ex]
		\hline
		1:0 & 3 & Failed in shear through notched end \tabularnewline [0.5ex]
		\hline
		1:0 & 4 & Flexural failure \tabularnewline [0.5ex]
		\hline
		
		1:2 & 1 & Flexural failure \tabularnewline [0.5ex]
		\hline
		1:2 & 2 & Flexural failure \tabularnewline [0.5ex]
		\hline
		1:2 & 3 & Failed in shear over opposing end to notch \tabularnewline [0.5ex]
		\hline
		1:2 & 4 & Failed in shear through opposing end to notch \tabularnewline [0.5ex]
		\hline
		
		1:4 & 1 & Failed in shear through existing rot, over notched end \tabularnewline [0.5ex]
		\hline
		1:4 & 2 & Flexural failure \tabularnewline [0.5ex]
		\hline
		1:4 & 3 & Flexural failure \tabularnewline [0.5ex]
		\hline
		1:4 & 4 & Flexural failure \tabularnewline [0.5ex]
		\hline
	\end{tabularx}
	\label{tab:Round_Fail}
\end{center}

\vspace*{\baselineskip}

\noindent
In most cases, the specimens followed the three modes in notch cracking as predicted; where a crack would first open and the notch corner (mode 1) and then shear (mode 2), as can be seen in Figure \ref{fig:Round_open}. 

\vspace*{\baselineskip}

\begin{figure}[h]
	\begin{center}
	\includegraphics[scale=0.39]{Roun_Crack_Open}
	\end{center}
	\caption{Specimen 2 notch 1:2, notch crack shearing and opening}
	\label{fig:Round_open}
\end{figure}

\pagebreak

\noindent
The notch crack would then gradually propagate to roughly the centre of the beam when the notch crack fully opens and flexural failure occurs, as shown in Figure \ref{fig:Round_Fail}.

\vspace*{\baselineskip}

\begin{figure}[h]
	\begin{center}
		\includegraphics[scale=0.4]{Round_Fail}
	\end{center}
	\caption{Specimen 4 notch 1:4, notch crack fully opening and ultimate failure}
	\label{fig:Round_Fail}
\end{figure}

\noindent
The notch crack propagation was found to be a lot more gradual compared to the rectangular specimens, which occurred almost instantaneously. Due to this, the time between the notch crack fully propagating and ultimate failure was shorter for the round specimens than the rectangular. 

\vspace*{\baselineskip}
\noindent
Specimen 1 of 1:0 slope notch failure was unexpected and different from any other specimen failure throughout this experiment. The specimen de-laminated from an existing knot, shown in Figure \ref{fig:Round_knot}, located on the small notched support section, over the roller support.
 
\vspace*{\baselineskip}

\begin{figure}[h]
	\begin{center}
		\includegraphics[scale=0.1]{Round_knot}
	\end{center}
	\caption{Specimen 1 notch 1:0, existing knot at small notched section over roller support}
	\label{fig:Round_knot}
\end{figure}

\noindent
This de-lamination gradually sheared through the grain, when the notch corner then cracked and also began to propagate. However, the notch crack did not propagate more than 5cm nor get to mode 3. Instead a shear crack began to form over the notch corner, which can be seen in Figure \ref{fig:Shear_knot}. Once the de-lamination crack propagated to about mid-way through the member, small shear cracks appeared along the specimen, the crack over the notch began to open, and the member ultimately failed in shear.  

\vspace*{\baselineskip}

\begin{figure}[h]
	\begin{center}
		\includegraphics[scale=0.4]{Shear_knot}
	\end{center}
	\caption{Specimen 1 notch 1:0, final failure}
	\label{fig:Shear_knot}
\end{figure}

\noindent
The shear failure that occurred in the remaining specimens followed the exact same process as seen for the rectangular specimens. Again, there was no obvious reason to the location of the shear failure, except for specimen 1 of slope 1:4, which sheared through existing rot. 

\subsubsection{Specimen Capacities}
The loads and times at which each specimen underwent notch failure and ultimate failure is shown in Table \ref{Tab:Round_Fail_2}. The results are even more variable throughout each notch slope experiment than the rectangular specimens. The reason for this is presumed to be due to the round specimens having significantly more defects than the rectangular specimens.

\vspace*{\baselineskip}

\captionof{table}{Round Specimen Crack Inititation and Ultimate Loads}
\begin{center}
	\begin{tabularx}{\textwidth}{|>{\centering}X|>{\centering}X|>{\centering}X|>{\centering}X|>{\centering}X|>{\centering}X|} 
		\hline
		
		\textbf{Notch Angle Slope} & \textbf{Specimen No.} & \textbf{Notch Crack Initiation (kN)} & \textbf{Time for Notch to Crack (min:sec)} & \textbf{Ultimate Failure (kN)} & \textbf{Time to Ultimate Failure (min:sec)} \tabularnewline [0.5ex] 
		\hline
		1:0 & 1 & 53.50 & 05:20 & 59.98 & 06:00 \tabularnewline [0.5ex]
		\hline
		1:0 & 2 & 44.46 & 04:25 & 60.06 & 05:59 \tabularnewline [0.5ex]
		\hline
		1:0 & 3 & 36.23 & 03:49 & 55.91 & 05:55 \tabularnewline [0.5ex]
		\hline
		1:0 & 4 & 22.76 & 02:15 & 36.62 & 03:38 \tabularnewline [0.5ex]
		\hline
		
		1:2 & 1 & 47.68 & 03:48 & 61.83 & 04:57 \tabularnewline [0.5ex]
		\hline
		1:2 & 2 & 47.05 & 04:41 & 53.42 & 05:19 \tabularnewline [0.5ex]
		\hline
		1:2 & 3 & 64.65 & 06:26 & 76.94 & 07:40 \tabularnewline [0.5ex]
		\hline
		1:2 & 4 & 38.92 & 03:50 & 42.99 & 04:18 \tabularnewline [0.5ex]
		\hline
		
		1:4 & 1 & 47.56 & 04:45 & 49.11 & 04:50 \tabularnewline [0.5ex]
		\hline
		1:4 & 2 & 58.48 & 05:46 & 58.48 & 05:46 \tabularnewline [0.5ex]
		\hline
		1:4 & 3 & 65.76 & 06:50 & 70.69 & 07:03 \tabularnewline [0.5ex]
		\hline
		1:4 & 4 & 69.01 & 06:52 & 69.01 & 06:52 \tabularnewline [0.5ex]
		\hline
		
	\end{tabularx}
	\label{Tab:Round_Fail_2}
\end{center}

\vspace*{\baselineskip}
\noindent
There is somewhat of a pattern between the water content and the and the failure capacities; where a higher content yields a lower capacity. This is prominent in the ultimate failure capacities for the 1:0 and 1:2 slope specimens, but the 1:4 slope data does not comply with this trend. As for the notch crack capacity, the same somewhat trend is present throughout the experiments, except for specimen 1 of notch slope 1:0, which failed in a different location prior to cracking at the notch corner. Although there is a slight pattern in the capacity in relation to the water contents, the amount of defects present during testing make it difficult to verify this trend. 

\vspace*{\baselineskip}

\begin{figure}[h]
	\begin{center}
		\fbox{\includegraphics[scale=0.8]{Round_Spec_Cap}}
	\end{center}
	\caption{Capacities for Circular Section Specimens}
	\label{fig:Round_Spec_Cap}
\end{figure}

\noindent
The capacities given in Table \ref{Tab:Round_Fail_2} were graphed for each notch slope and displayed in Figure \ref{fig:Round_Spec_Cap}, where the bars indicated the average capacities and the error bars show the variance in the results. From this graph, it is observed that that the ultimate capacity increases with an increasing notch slope. The difference between the average ultimate capacities for the 1:0 and 1:2 slope specimens is slightly greater than that for the 1:2 and 1:4 specimens, being 5.6kN and 3.0kN respectively. This is most likely due to the two of the 1:0 slope specimens be loaded at a slightly slower rate, which effect is unknown on the capacity. A more distinct increase and trend is observed for average notch capacity, through the increasing notch slopes; with a difference of 10.3kN and 10.6kN respectively. Overall, it is determined that a 1:0 slope notch yields the lowest capacities and thus is the critical notch angle for the round specimens.

\vspace*{\baselineskip}
\noindent
Figure \ref{fig:Round-Crack-fail} shows the time between the crack initiating at the notch corner and ultimate failure for the round specimens.
\vspace*{\baselineskip}

\begin{figure}[h]
	\begin{center}
		\fbox{\includegraphics[scale=0.8]{Round_crack_fail}}
	\end{center}
	\caption{Time between notch crack initiation and ultimate failure}
	\label{fig:Round-Crack-fail}
\end{figure}

From this graph, a distinct trend can be observed, where the time between the notch cracking and ultimate failure significantly decreases with an increasing notch slope. This was not unexpected, as it appears from previous results that the increase in notch slope increases the capacity of the notch, thus the load at which the notch cracks is getting closer to the ultimate capacity of the beam. The same issue arises as discussed for the rectangular sections, with considering current maintenance standards and the small amount of time between crack initiation and ultimate failure for a 1:4 slope notch.

\subsubsection{Strain Gauge Analysis}
Due to the curvature of the specimens, the strain gauges experienced some initial strain before loading began. To account for this, the strain gauges were zeroed before loading commenced (i.e.at a load of zero).

\pagebreak

\noindent
\textbf{Notch Slope 1:0}\par
\noindent
The load-displacement data collected from the LVDT is given in Figure \ref{fig:Round_10_def} for all four round specimens with a notch slope of 1:0.

\vspace*{\baselineskip}

\begin{figure}[h]
	\begin{center}
		\fbox{\includegraphics[scale=0.8]{Round_10_def}}
	\end{center}
	\caption{Load-displacement graph for round beams with notches of slope 1:0}
	\label{fig:Round_10_def}
\end{figure}

\noindent
The load-displacement graphs for specimens 1, 2 and 3 roughly follow the trend of a standard load-displacement graph, except specimen 1 becomes erratic at around 42kN load. This odd behaviour displayed by specimen 1 is presumed to be correlating with the initiation of the de-lamination from the existing knot as discussed earlier. The load-displacement for specimen 4 does not follow the standard trend. The reason for this is unknown, but may be related to an issue with the LVDT or an existing defect. Overall, these graphs appear to be significantly smoother than those for the rectangular specimens, which is assumed to be due to the notch cracks propagating more gradually in the round specimens than the rectangular.

\vspace*{\baselineskip}

\noindent
The compressive strain at the top and tensile strain at the bottom in the centre of the beam is shown in Figure \ref{fig:Round_10_Centre}, where it was graphed against the applied load. Specimens 2, 3 and 4 experience an increasing linear trend in both compression and tension until they suddenly plateau. 

\vspace*{\baselineskip}

\begin{figure}[h]
	\begin{center}
		\fbox{\includegraphics[scale=0.8]{Round_10_Centre}}
	\end{center}
	\caption{Top and bottom strain at centre for round Beams with notches of slope 1:0}
	\label{fig:Round_10_Centre}
\end{figure}
\pagebreak
\noindent
The point at which the graph plateaus is assumed to be cause by the notch crack fully propagating. Specimen 1 follows slightly different trends in both compression and tension, which is most likely due to the initial crack occurring due to the existing defect, on the opposing end to the notch.

\vspace*{\baselineskip}

\noindent
The average horizontal strain across the section at the notch corner with relation to the applied load in shown in Figure \ref{fig:Round_10_Z}. It is observed that all specimens horizontal strains linearly increase until they reach a point where they begin to curve, except for specimen 1 which becomes more erratic. For specimens 2 and 3, this point aligns with the notch crack initiating, and for specimen 1 it is assumed that this is the point the crack initiated from the existing defect. 

\vspace*{\baselineskip}
\begin{figure}[h]
	\begin{center}
		\fbox{\includegraphics[scale=0.7]{Round_10_Z}}
	\end{center}
	\caption{Horizontal strain for round beams with notches of slope 1:0}
	\label{fig:Round_10_Z}
\end{figure}
\pagebreak

\noindent
As for specimen 4, the point at which the graph curves occurs after the crack had initiated and is assumed to be the point at which the crack fully propagates, which coincides with the graph in Figure \ref{fig:Round_10_Centre}. The graph for specimen 4 then becomes erratic, which is most likely due to an existing crack, located along the bottom of the notched section and can be seen in Figure \ref{fig:Spec_4}.

\vspace*{\baselineskip}

\begin{figure}[h]
	\begin{center}
		\includegraphics[scale=0.09]{Spec_4}
	\end{center}
	\caption{Round specimen 4 of 1:0 notch slope, existing defect}
	\label{fig:Spec_4}
\end{figure}

\noindent
The average vertical strain across the section at the notch corner with relation to the applied load in shown in Figure \ref{fig:Round_10_Y}. It can be seen that all graphs follow roughly the same pattern; where they linearly increase, begin to plateau and then change direction. Unlike the rectangular specimens, there does not seem to be a clear reason for this trend and it is hard to distinguish where the notch crack initiated in these results. 

\vspace*{\baselineskip}

\begin{figure}[h]
	\begin{center}
		\fbox{\includegraphics[scale=0.8]{Round_10_Y}}
	\end{center}
	\caption{Vertical Strain for round beams with notches of slope 1:0}
	\label{fig:Round_10_Y}
\end{figure}

\vspace*{\baselineskip}

\noindent
\textbf{Notch Slope 1:2}\par
\noindent
The load-displacement graphs for all specimens with a 1:2 slope notch roughly followed the standard trend, see Appendix C. The compressive and tensile strains at the centre of the beam can be seen in Figure \ref{fig:Round_12_Centre}. This graph shows the compressive and tensile strain experienced by the 1:2 slope specimens had a pattern similar to that experienced by the rectangular specimens; with a more distinct decrease in strain. This decrease correlates to the point at which the notch crack initiates. The graph from Figure \ref{fig:Round_12_Centre} indicates the notch crack initiation was more abrupt in the 1:2 slope specimens than was in the 1:0 slope specimens.

\vspace*{\baselineskip}

\begin{figure}[h]
	\begin{center}
		\fbox{\includegraphics[scale=0.8]{Round_12_Centre}}
	\end{center}
	\caption{Top and bottom strain at centre for round beams with notches of slope 1:2}
	\label{fig:Round_12_Centre}
\end{figure}
\pagebreak

\noindent
The horizontal and vertical strain for the 1:2 slope specimens are shown in Figure \ref{fig:Round_12_ZY}. Both these graphs initially follow the same linearly increasing trend until they hit a point and the graphs change. The horizontal strains experience a decrease or plateau in load after this point, except for specimen 1 which becomes erratic, which assumed to be due to an existing defect. Whereas the vertical strains suddenly decrease in strain after the point where they stop being linear. This point correlates to the load at which the notch crack initiates and supports the earilier statement that the crack initiation is more abrupt than for the 1:0 slope sound specimens. 

\vspace*{\baselineskip}

\begin{figure}[h]
	\begin{center}
		\fbox{\includegraphics[scale=0.6]{Round_12_Z}}
		\fbox{\includegraphics[scale=0.6]{Round_12_Y}}
	\end{center}
	\caption{Horizontal (top) and vertical (bottom) strain for round beams with notches of slope 1:2}
	\label{fig:Round_12_ZY}
\end{figure}
\pagebreak

\noindent
\textbf{Notch Slope 1:4}\par
\noindent
The load-displacement graph for specimens 2, 3 and 4 of 1:4 slope notch round specimens followed the standard trend, but specimen 1 followed the same trend as experienced by specimen 4 of 1:0 slope, see Appendix C. This difference in trend for specimen 1 is indicative that it had a defect.

\vspace*{\baselineskip}
\noindent
The compressive and tensile strain experienced by all specimens followed the exact same trend as experienced by the 1:2 slope round specimens; where the sudden drop in the trends align with the load at which the notch crack initiates. See Appendix C for slope 1:4 round specimen compressive and tensile strain at beam centre graph.

\begin{figure}[h]
	\begin{center}
		\fbox{\includegraphics[scale=0.6]{Round_14_Z}}
		\fbox{\includegraphics[scale=0.6]{Round_14_Y}}
	\end{center}
	\caption{Horizontal (top) and vertical (bottom) strain for round beams with notches of slope 1:4}
	\label{fig:Round_14_ZY}
\end{figure}

\noindent
The average horizontal and vertical strains experienced by the 1:4 round specimens through the notch corner can be seen in Figure \ref{fig:Round_14_ZY}. Specimens 2, 3 and 4 follow the same linear trend in both horizontal and vertical strains, until they hit a point where the graph suddenly changes. For specimens 2 and 4, the notch crack initiated almost exactly at the point of failure, where specimen 2's strain gauges cut out, the strain of specimen 4 captures the point of notch failure. Specimen 3 suddenly decreases in load once the graph stops being linear. There is no distinct reason for this sudden drop, however it is assumed to be due to the notch crack fully propagating. Specimen 1 experiences significantly different strain trends in both horizontal and vertical strains, this supportive of the earlier statement that specimen 1 had a defect which affected the strain gauge results.    

\subsection{Comparison of Round and Rectangular Section}
The main difference observed between the circular and rectangular section specimens was the general ultimate failure mode for the round specimens was flexural failure, where the rectangular members experienced shear failure. The reason for this is unknown, but is presumed to be due to the way the grain responds to the load between the two different cross-section. 
\vspace*{\baselineskip}

\noindent
Another difference that was noticed between the two sections was the notch propagation for the rectangular section happened almost instantaneously, whereas the notch crack tended to propagate very gradually through the circular section specimens. This is also presumed to be due the two different cross-sections respond to the load. 

\vspace*{\baselineskip}

\noindent
The graph presented in Figure \ref{fig:compare_cap} shows the average ultimate capacities graphed with respect to the angle of the notch chamfer, from the notch corner. It is observed that the overall trend for both sections is very similar, with the round specimens yielding a slightly greater initial incline than the rectangular members. However, this may be due to the circular section having a greater response to the increase in slope angle, or it could be  due to two of the 1:0 slope specimens being loaded at a slightly slower rate.

\vspace*{\baselineskip}

\begin{figure}[h]
	\begin{center}
		\fbox{\includegraphics[scale=0.67]{Com_Cap}}
	\end{center}
	\caption{Comparison of rectangular and circular section ultimate capacities}
	\label{fig:compare_cap}
\end{figure}

\noindent
The average notch capacities in relation to the angle of the slope from the notch corner can be seen in Figure \ref{fig:compare_not}. Both the rectangular and round specimens yield similar trends, and similarly the incline between the 1:0 and 1:2 slope notch is greater for the round than the rectangular specimens. However, the notch capacities for the rectangular specimens seem to respond more to the increase in notch angle than the round specimens. The reason for this is unknown, but is presumed to be due the they way the two section types respond to the load.

\vspace*{\baselineskip}

\begin{figure}[h]
	\begin{center}
		\fbox{\includegraphics[scale=0.65]{Comp_Notch}}
	\end{center}
	\caption{Comparison of rectangular and circular section notch capacities}
	\label{fig:compare_not}
\end{figure}

\subsection{Predicted Capacities using Standards}

\subsubsection{AS1720 Notch Design}
Using AS1720 design equation for rectangular section specimens, the graph shown in Figure \ref{fig:AS_notch} was established; which compares the design capacities predicted and the actual capacities of each notch slope. The blue solid bars show the estimated capacities of the notches using a $f'_{sj}$ of 4.2 for an S3 grade timber, and the error bars show the variance in the estimated results ranging from S2 to S5 grade timber. The safety factor ($\phi$) was not incorporated in calculating the notch capacities. The red bars show the average results, calculated using the left hand side of the design Equation \ref{eq:Aus}, with the $M*$ and $V*$ occurring at the notch corner in relation to the load at which the notch cracks (the notch capacity) for the rectangular specimens.

\begin{figure}[h]
	\begin{center}
		\fbox{\includegraphics[scale=0.7]{AS_notch}}
	\end{center}
	\caption{Comparison of AS1720 notch design capacities with experimental results}
	\label{fig:AS_notch}
\end{figure}

\vspace*{\baselineskip}

\noindent
From this graph, it is observed that the predicted capacities increase with an increasing notch angle as do the actual notch capacities. However, there is an order of magnitude difference between the theoretical and actual notch capacities, theoretical to actual capacity ratios of 0.12, 0.19 and 0.19 for a 1:0, 1:2 and 1:4 slope notch respectively. Even though this method accounts for the increase in notch capacity through the increasing notch slope, from these results, this method is deemed inaccurate at estimating the capacity of the notch. The reason this design method yielded unsatisfactory results is assumed to be due to the assumption that timber is linearly elastic, made in derivation for this method.  A further limitation of this method is it is purely designed for a rectangular section specimen and there is no obvious way to alter the design equation to accommodate a different beam section.

\subsubsection{Eurocode 5 Notch Design}
The standard European notch design equation in Eurocode 5, see Equation \ref{eq:Euro}, was used to determine a theoretical shear stress at the notch corner as per the right hand side of the equation, for the rectangular specimens. This can be seen in Figure \ref{fig:Eurocode}, where the theoretical capacity determined is compared to the actual shear stress ($\tau_{d}$) at which the notch crack initiated. The blue bar in the graph represents the theoretical capacity, calculated using a timber strength class D40 for Eucalyptus trees. The corresponding error bars show the range in results using a D60 to D30 class, which correlate with Australian standard S2 to S5 classes. The red bars show the average shear stress at the notch corner and the error bars show the variance in the experimental results. 

\begin{figure}[h]
	\begin{center}
		\fbox{\includegraphics[scale=0.7]{Eurocode}}
	\end{center}
	\caption{Comparison of Eurocode 5 notch design capacities with experimental results}
	\label{fig:Eurocode}
\end{figure}

It can be observed from Figure \ref{fig:Eurocode} that theoretical results and actual results follow a similar trend, where the capacity increases with an increasing notch slope. The ratio of the actual and theoretical results are determined to be 0.45, 0.54 and 0.51 for a 1:0, 1:2 and 1:4 notch slope respectively. This ratio shows the Eurocode notch design method is relatively accurate, with a factor of safety of around 0.5 for all notch slopes. The only limitation to this method is it only designs for a rectangular section specimen. 

\subsubsection{CSA O.86 Notch Design}
The graph shown in Figure \ref{fig:CSA} was established to compare the CSA O.86 notch design method to the actual results gained through experimentation. The blue bar shows the calculated notch resistance for sawn lumber under wet service conditions, excluding the safety factor ($\Phi$). This resistance can be compared to the average shear force at the notch corner, relating to the load at which the notch crack to initiated, shown by the red bar. It can be seen from this graph that this design method is inaccurate, having a very low theoretical to actual capacity ratio of 0.17 for a rectangular specimen of 1:0 notch slope. This method also has a major limitation, where it does not account for a change in notch slope and only designs for a rectangular section. If this same capacity is compared to the 1:2 and 1:4 notch slope results, a theoretical to actual ratio of 0.16 and 0.11 are achieved respectively. Thus it is determined that this method was not accurate in calculating the notch capacity for the tested specimens.

\vspace*{\baselineskip}

\begin{figure}[h]
	\begin{center}
		\fbox{\includegraphics[scale=0.7]{CSA}}
	\end{center}
	\caption{Comparison of CSA O.86 notch design capacities with experimental results}
	\label{fig:CSA}
\end{figure}

\subsubsection{AS1720 Ultimate Section Design}
\textbf{Shear}
For a circular section, the equation to determine the ultimate design shear load ($V_{d}$) is shown in Equation \ref{eq:V}
	\begin{equation}
	V_{d} = \phi k_{1} k_{4} k_{6} k_{20} f'_{s} A_{s}
	\label{eq:V}
	\end{equation}
\noindent
and the same equation applies for a rectangular section, accept it does not incorporate the factor $k_{20}$. After considering the shear equation given for both the rectangular and circular section, as well as general equations in determining shear force, it was established that the shear area can be found using, 
	
	\begin{equation}
	A_{s} = \frac{It}{Q}
	\label{eq:As}
	\end{equation}

where \par
$ I $ = Moment of Inertia\par
$ Q $ = Statical Moment of Area \par
$ t $ = Thickness in Material Perpendicular to Shear \par

\vspace*{\baselineskip}

\noindent
With this generalised equation for the shear area, an ultimate shear design load can be established for any cross-section. Thus, a design method can be established for a end notched specimen, using the cross-section over the notch. The following equations were established for the truncated, circular section over the notched end.

	\begin{equation}
	I_{notch} = \pi R^{2}\bigg[\bigg(\frac{4R}{3\pi}\bigg)^{2}+k^{2}\bigg]- \frac{R^{4}}{8}\bigg[m+2\sin(2\theta)\sin^{2}(\theta)\bigg]+\frac{R^{2}mk}{2}\bigg[\frac{8R\sin^{3}(\theta)}{3m}-k^{2}\bigg]+2R^{4}\bigg[\frac{\pi}{8}-\frac{8}{9\pi}\bigg]
	\label{eq:Itot}
	\end{equation} 
	
	where\par
	$ I_{notch} $ = Depth of notch \par
	$ d $ = Depth of notch \par
	$ R $ = Radius \par
	$ k  = R - d$ \par
	$ \theta = \arccos(k/R)$ \par
	$ m = 2\theta-\sin(2\theta)$ \par

	\vspace*{\baselineskip}
	\noindent
	The thickness in perpendicular to the shear load ($t$) and the statical moment of area ($Q_{notch}$) for the notched area are as follows, 
	\begin{equation}
	t = 2\sqrt{R^{2}-( \bar y_{notch}-k^{2})}
	\end{equation}
	
	\begin{equation}
	Q_{notch} = A_{notch} \times \bar y_{notch}
	\end{equation}
	
	\noindent
	with a total cross sectional area above the notch ($A_{notch}$) and vertical centroid ($\bar{y}_{notch}$) are given by
	
	\begin{equation}
	A_{notch} = \frac{R^{2}}{2}(2\pi-m)
	\end{equation}
	
	\begin{equation}
	\bar y_{tot} = k + \frac{4R}{3(2\pi-m)}\sin^{3}(\theta)
	\end{equation}

\noindent
Using these equations in conjunction with Equations \ref{eq:As} and \ref{eq:V}, an ultimate design shear load was established for the specific specimen, of 100mm diameter and 25mm deep notch, to be approximately 20.0kN, without a factor of safety ($\phi$). All k values were taken as 1.0 accept $k_{20}$ which was taken as 0.9, and $f'_{s}$ was 5.0 MPa for F22 grade timber. Similarly, using the standard design equation specified for a rectangular section, a design shear load was determined for the rectangular notched section to be 14.0kN, not factored for safety. This was calculated for a 60mm width and 70mm depth over the notch, using the same k and $f'_{s}$ as for the circular notched section. Figure \ref{fig:Shear_Des} shows a comparison between the theoretical shear capacities calculated using this design method, with the actual ultimate shear capacities found through testing.

\vspace*{\baselineskip}

\begin{figure}[h]
	\begin{center}
		\fbox{\includegraphics[scale=0.7]{Shear_Des}}
	\end{center}
	\caption{AS1720 ultimate shear design capacities compared to experimental results}
	\label{fig:Shear_Des}
\end{figure}

\noindent
The red bars represent the average shear capacity, with the error bars showing the variance in the results, and the blue bars show the calculated capacity for F22 grade timber and the error bars shows the variance for S2 and F14 grades. Figure \ref{fig:Shear_Des} shows the calculated capacities are relatively accurate, showing a similar variance in error between the theoretical and actual results, and an average difference in capacities of 6.5kN for the round specimens and 4.9kN for the rectangular. From these results, a ratio of the theoretical and actual results was determined to be 0.75 for the round specimens and 0.74 for the rectangular specimens, which shows this method gives relatively accurate design capacities. Although, there is an issue surrounding this method, as it technically does not account for a notch with a sloped chamfer and only design for a 1:0 slope. However, when the same capacity is compared to the 1:2 and 1:4 shear capacities, a ratio of 0.68 and 0.65 respectively for a round specimen, and 0.72 and 0.68 respectively for a rectangular specimen. These theoretical to actual capacity ratios show the design capacities are still fairly accurate in accounting for a 1:2 and 1:4 chamfer in both sections. Overall this method was fairly accurate in designing for both specimens and all notch slopes, even though it is not modified to account for the extra strength supplied by a chamfer. 

\vspace*{\baselineskip}
\noindent
\textbf{Bending}
\noindent
For a circular section, the equation to determine the ultimate design Moment ($M_{d}$) is shown in Equation \ref{eq:M}
\begin{equation}
M_{d} = \phi k_{1} k_{4} k_{6} k_{9} k_{12} k_{20} k_{21} k_{22} f'_{b} Z
\label{eq:M}
\end{equation}
\noindent
and the same equation applies for a rectangular section, accept it does not incorporate the factor $k_{20}$ to $k_{22}$. To establish a method in designing for the ultimate bending moment for a circular notched beam, the section modulus for the truncated notched section can be used in the design equation. Equation \ref{eq:Z} gives the section modulus for the notched circular section. 

	\begin{equation}
	Z = \frac{R^{4}}{R+k+\bar{y}_{tot}}\bigg(\alpha-\sin(\alpha)\cos(\alpha)+\sin^{3}(\alpha)\cos(\alpha)-\frac{16\sin^{6}(\alpha)}{9(\alpha-\sin(\alpha)\cos(\alpha))}\bigg)
	\label{eq:Z}
	\end{equation}
\noindent
Using these equations, a design bending capacity was established for the circular specimen as well as the rectangular specimen. A comparison of these estimated capacities to the experimental results can be seen in Figure \ref{fig:Bend_Des}.

\vspace*{\baselineskip}

\begin{figure}[h]
	\begin{center}
		\fbox{\includegraphics[scale=0.7]{Bend_Des}}
	\end{center}
	\caption{AS1720 ultimate bending design capacities compared to experimental results}
	\label{fig:Bend_Des}
\end{figure}

\noindent
The blue bars in Figure \ref{fig:Bend_Des} represent the capacities calculated using the standard moment design for an F22 grade timber, without incorporating a factor of safety ($\phi$). The corresponding error bars show the variance in the estimated capacities for a range from F27 to F14 grade timber. Overall this graph shows the calculated bending capacities using this method are fairly accurate, with a theoretical to actual capacity ratio of 0.53 for the round specimens and 0.59 for the rectangular. This method does have the same limitation as the shear, where it is only considering the truncated cross-section and does not account for a notch chamfer. However, when compared to the average experimental results for a 1:2 and 1:4 notch, the same calculated capacity achieves a ratio of 0.48 and 0.46 respectively, for the round specimen, and 0.58 and 0.55 respectively for the rectangular specimen. Thus the bending capacity is still fairly accurate for all notched specimens, although not as precise as the design shear capacity. 

\subsubsection{Comparison of Design Methods}
To compare the overall design methods, the ratios of the theoretical capacities to the actual capacities for each method was graphed, shown in Figure \ref{fig:Des_Comp}. The closer these ratios are to 1.0, the more accurate the method is. All the results shown in this graph are for the rectangular specimen only, and the same theoretical capacity is compared to all notch slopes for the methods that don't design account for them. 

\vspace*{\baselineskip}

\begin{figure}[h]
	\begin{center}
		\fbox{\includegraphics[scale=0.65]{Des_Comp}}
	\end{center}
	\caption{AS1720 ultimate bending design capacities compared to experimental results}
	\label{fig:Des_Comp}
\end{figure}

\noindent
When compared, it is deduced that AS1720 notch design and CSA O.86 yielded inaccurate results for calculating the experimental capacities, obtaining below 20\% accuracy. Whereas Eurocode achieved roughly 50\% accuracy and AS1720 ultimate design obtained around 70\% accuracy for shear and below 60\% for bending. The reason AS1720 notch design yielded unsatisfactory results is most likely due to the method's assumption that timber is a linearly elastic material. As for CSA O.86, the reason this method obtained inaccurate results is unknown, as it is based on the same fracture mechanics equation, derived by Gustafsson, as Eurocode design. However, when the design equations are compared, it can  be seen that CSA design significantly modifies Gustafsson's original equation, whereas Eurocode does not. Overall, the optimal design method in accordance with these experimental results was AS1720's ultimate design for shear and bending, using the cross section over the notch. However, as this method does not account for the a notch slope and is purely based on ultimate failure, it is determined that Eurocade 5 design was optimum in designing for notch failure for rectangular specimens.

\vspace*{\baselineskip}

\noindent
The only design method that designs for a notched round specimen is the modified AS1720 ultimate design in shear and bending, which achieved an accuracy of 75\% for shear and 53\% for bending.  

\pagebreak	

\section{Conclusions}

	\begin{enumerate}
		\item The critical notch slope, which yielded the lowest member capacities, was found to be a slope of 1:0 for botch rectangular and round specimens. 
		\item A notch slope of 1:4 was found to obtain the highest member capacities for both sections, complying with recommendations from AS1720 and Department of Transport and Main Roads to use this notch slope in timber bridges.
		\item The effects of short term loading showed the time between crack initiation and ultimate failure reduced significantly with an increasing notch slope. These results warrant further investigation into the effects of long-term loading on this time-frame, to determine if current maintenance standards require modification to account for this.
		\item Notch crack growth was found to be more brittle and sudden in the rectangular specimens than the round; which showed a more gradual crack growth.
		\item The optimal design method was determined to be AS1720 ultimate design for shear and bending, using section above the notch. This method was modified to account for the circular section, thus obtaining a method for designing notched round members. 
	\end{enumerate}

\pagebreak	

\bibliographystyle{unsrt}
\bibliography{My_Library.bib}

\pagebreak
\cleardoublepage
\pagenumbering{gobble}

\appendixtitleon

\begin{appendices}
	\section{\textit{Rectangular Specimen Characteristics}}
\pagebreak

\end{appendices}

\begin{appendices}
	\section{\textit{Round Specimen Characteristics}}
	\pagebreak
	
\end{appendices}

\begin{appendices}
	\section{\textit{Strain Gauge Results}}
	\pagebreak
	
\end{appendices}

\end{document}
